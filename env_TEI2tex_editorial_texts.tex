\startenvironment env_TEI2tex_editorial_texts % Die TEI nach TeX Konvertierung
\project project_ediarum_context

\setuphead[section][number=no,width=0.95\textwidth,style=\bfa,]
\setupalign[hyphenated,morehyphenation]
\hyphenchar\font=\string"7F

%% ##################################################################
%% ##                      Basiselemente                           ##
%% ##################################################################

%% Generelle Registrierung des xml-Markups
%%\setupxml[default=none]
\startxmlsetups xml:tei:setups
	%% Alle nicht explizit definierten Elemente bleiben unberücksichtigt.
	\xmlsetsetup{tei}{*}{-}
	%% Der Startpunkt wird registriert.
	\xmlsetsetup{tei}{TEI|title|apos}{xml:tei:*}
	%% Die Textstrukturelemente folgen.
	\xmlsetsetup{tei}{text|body|div}{xml:tei:*}
	%% Die Absatzelemente sind:
	\xmlsetsetup{tei}{p|head|hi}{xml:tei:*}
	%% Nun folgen die einfachen Textelemente.
	\xmlsetsetup{tei}{bibl|hi|persName|placeName|lb}{xml:tei:*}%TODO Sachkommentare
	%% Als nächstes kommen alle Elemente mit Apparateintrag.
	\xmlsetsetup{tei}{note}{xml:tei:*}%TODO Sachkommentare
	%% Dieses sind die Elemente für Register und Verweise
	\xmlsetsetup{tei}{anchor|index}{xml:tei:*}%TODO Sachkommentare
\stopxmlsetups

%% Das Register wird unter 'tei' registriert.
\xmlregisterdocumentsetup{tei}{xml:tei:setups}

%% ##################################################################
%% ##               Der Einstieg in das Dokument                   ##
%% ##################################################################

%% Der Brief wird generell vorbereitet und die Metadaten werden ausgelesen.
\startxmlsetups xml:tei:TEI
	%% Falls die Verarbeitung von externen Referenzen stattfinden soll, werden die Briefdaten extern gespeichert, ..
	\doifmode{FSBreferenzen}
		{\FSBfwriteT%
		%% .. und zwar mit Briefnummer ..
		\xmldoifelsetext{#1}{./teiHeader/fileDesc/titleStmt/title}
			{\FSBfwrite{<file xml:id="\xmlatt{#1}{xml:id}" type="Brief"}%
			\FSBfwrite{ idno="\xmltext{#1}{./teiHeader/fileDesc/titleStmt/title}">}%
			}
			%% .. oder ohne.
			{\FSBfwrite{<file xml:id="\xmlatt{#1}{xml:id}" type="Brief"}%
			\FSBfwrite{ idno="\xmlatt{#1}{xml:id}">}%
			}%
		\FSBfwriteN%
		}
	%% Falls ein Deckblatt für einen einzelnen Brief erzeugt werden soll, werden dort die Metadaten hingeschrieben.
	\doifmode{deckblatt}
		{\xmlcommand{#1}{.}{xml:tei:deckblatt}}%
	%% Hier beginnt die Verarbeitung für den Druck, ..
	%% .. dabei wird entschieden, ob das Regest erscheint oder der Editionstext.
	\xmldoifelse{#1}{div[@type='nurRegest']}
		{% Falls nur das Regest erscheinen soll, ..
		%% .. werden nur die Briefnummer und der Titel ..
		\bgroup
		\switchtobodyfont[12pt]
		\it
		%% .. als Markierung für die Kopfzeile ..
		\xmldoifelse{#1}{./teiHeader/fileDesc/titleStmt/title}
			{\marking[FSBbriefnummer]{\xmltext{#1}{./teiHeader/fileDesc/titleStmt/title}}
			}
			{\marking[FSBbriefnummer]{\xmlatt{#1}{xml:id}}
			}
		\marking[FSBbriefdatum]{\xmltext{#1}{./teiHeader/profileDesc/creation/date[@type='erstellung']}}
		%% .. und als Überschrift verwendet.
		%%\xmldoifelse{#1}{./teiHeader/fileDesc/titleStmt/title/idno}
		%%	{*\xmltext{#1}{./teiHeader/fileDesc/titleStmt/title/idno}.\hskip2mm
		%%	}
		%%	{*\xmlatt{#1}{xml:id}.\hskip2mm
		%%	}
		\xmltext{#1}{./teiHeader/fileDesc/titleStmt/title}
		\egroup
		%% Dann erscheint das Regest im Text ..
		\blank[big]
		\bgroup\it\xmltext{#1}{./teiHeader/fileDesc/sourceDesc/msDesc/msContents/summary}\egroup
		%% .. und schließlich noch die zugehörige Bemerkung in der Fußnote.
		\FSBdotextapparat{%
			\bgroup\bi *\xmltext{#1}{./teiHeader/fileDesc/titleStmt/title/idno}.\egroup
			\hskip5mm
			\bgroup\it\xmltext{#1}{./teiHeader/fileDesc/sourceDesc/msDesc/msContents/p/note}\egroup
			}
		\blank[2*big]
		}
		{% Im Normalfall wird der Brief mit Überlieferung und Editionstext gedruckt.
		%% Auch dann werden die Briefnummer und der Titel ..
		\bgroup
		\switchtobodyfont[12pt]
		\it
		%% .. als Markierung für die Kopfzeile ..
		\xmldoifelse{#1}{./teiHeader/fileDesc/titleStmt/title}
			{\marking[FSBbriefnummer]{\xmltext{#1}{./teiHeader/fileDesc/titleStmt/title}}
			}
			{\marking[FSBbriefnummer]{\xmlatt{#1}{xml:id}}
			}
		\marking[FSBbriefdatum]{\xmltext{#1}{./teiHeader/profileDesc/creation/date[@type='erstellung']}}
		%% .. und als Überschrift verwendet.
		%%\xmldoifelse{#1}{./teiHeader/fileDesc/titleStmt/title/idno}
		%%	{\xmltext{#1}{./teiHeader/fileDesc/titleStmt/title/idno}.\hskip2mm
		%%	}
		%%	{\xmlatt{#1}{xml:id}.\hskip2mm
		%%	}
		%%\xmltext{#1}{./teiHeader/fileDesc/titleStmt/title}
		\xmlall{#1}{./teiHeader/fileDesc/titleStmt/title}
		\egroup
		%% Schließlich wird der Editionstext selbst gedruckt.
		\xmlall{#1}{./text}
		\blank[2*big]
		}%
	%% Falls die Referenzen exportiert wurden, wird das entsprechend abgeschlossen.
	\doifmode{FSBreferenzen}{\FSBfwriteT\FSBfwrite{</file>}\FSBfwriteN}
\stopxmlsetups

\startxmlsetups xml:tei:title
	\xmlflush{#1}
\stopxmlsetups


\startxmlsetups xml:tei:apos
	\'{}
\stopxmlsetups

%% ##################################################################
%% ##              Die Strukturelemente des Textes                 ##
%% ##################################################################

%% Der Einstieg in die Ausgabe des edierten Textes
\startxmlsetups xml:tei:text
	%% Zuerst wird die Zeilennummerierung auf null gesetzt, ..
	\FSBresetlinenumbering[FSBdefaultlinenumbering]
	%%  .. dann wird der Inhalt ausgegeben, ..
	\xmlall{#1}{./body}
	%% .. und zum Schluss die vorhandenen Autorenfussnoten gedruckt.
	\placenotes[FSBautornote]
\stopxmlsetups

%% Im Textkörper tauchen nur div auf.
\startxmlsetups xml:tei:body
	\xmlall{#1}{./div}
\stopxmlsetups

%% Jeder Textbereich kann p enthalten.
\startxmlsetups xml:tei:div
	\xmlall{#1}{p|head}
\stopxmlsetups


%% ##################################################################
%% ##                   Die Absatzelemente                         ##
%% ##################################################################

\startxmlsetups xml:tei:head
	\par
	\doif{\xmlatt{#1}{type}}{h1}{
		\bgroup
		\bfc
		\xmlflush{#1}
		\egroup
	}
	\doif{\xmlatt{#1}{type}}{h2}{
		\bgroup
		\bfb
		\xmlflush{#1}
		\egroup
	}
	\doif{\xmlatt{#1}{type}}{h3}{
		\bgroup
		\bfa
		\section{\xmlflush{#1}}
		\egroup
	}%
\stopxmlsetups

%\startxmlsetups xml:tei:head:hi
%	
%	\doifinstring{-}{\expand{\xmltext{#1}{}}}{
%		haha
%	}
%\stopxmlsetups

% Die normalen Absätze enthalten den meisten Text.
\startxmlsetups xml:tei:p
	%% Zuerst beginnt der neue Paragraph
	\par
	%% Falls der Absatz zentriert erscheinen soll, ..
	\doifelse{\xmlatt{#1}{rend}}{align(center)}
		{% .. wird er entsprechend gesetzt ..
		\bgroup
		\startalignment[middle]
		%% .. mit Zeilennummerierung, ..
		\startlinenumbering[FSBdefaultlinenumbering][continue]
		\xmlflush{#1}%TODO Sachkommentare
		\stoplinenumbering
		\stopalignment
		\egroup
		}
		{% .. ansonsten bleibt er auch mit Zeilennummerierung linksbündig.
		\startlinenumbering[FSBdefaultlinenumbering][continue]
		\xmlstripnolines{#1}{.}
		\xmlflush{#1}%TODO Sachkommentare
		\stoplinenumbering
		}%
\stopxmlsetups

%% ##################################################################
%% ##                Die einfachen Textelemente                    ##
%% ##################################################################

%% Gekennzeichnete Werke werden nur ausgegeben.
\startxmlsetups xml:tei:bibl
%	\ignorespaces\xmlflush{#1}\removeunwantedspaces
	\xmlstrip{#1}{.}
	\xmlflush{#1}
\stopxmlsetups

%% Hervorgehobene Stellen werden gesperrt gedruckt.
\startxmlsetups xml:tei:hi
	\xmlstrip{#1}{.}
	\doifinstring{b}{\xmlatt{#1}{rendition}}{
		\bgroup
		\bf
		\xmlflush{#1}
		\egroup
	}
	\doifinstring{i}{\xmlatt{#1}{rendition}}{
		\bgroup
		\it
		\xmlflush{#1}
		\egroup
	}%
\stopxmlsetups

%% Personen werden nur ausgegeben.
\startxmlsetups xml:tei:persName
	%%\xmlstrip{#1}{.}
	\xmlflush{#1}
\stopxmlsetups

%% Auch Orte werden nur ausgegeben.
\startxmlsetups xml:tei:placeName
	\xmlstrip{#1}{.}
	%%\doifinstring{-}{\xmltext{#1}{.}}{
	%%	\expanded{\LUAreplacedash{\xmltext{#1}{.}}}
	%%}
	\xmlflush{#1}
\stopxmlsetups

\startxmlsetups xml:tei:lb
	\hfil\break
\stopxmlsetups

%% ##################################################################
%% ##             Alle Elemente mit Apparateintrag                 ##
%% ##################################################################

%% Eine Autorenfussnote wird als Endnote im Text gedruckt.
\startxmlsetups xml:tei:note
	\footnote{\xmlflush{#1}}
\stopxmlsetups

%% ##################################################################
%% ##                      Register und Verweise                   ##
%% ##################################################################

%% Bei den Ankern werden, wenn gewünscht, die Zeilennummern exportiert.
\startxmlsetups xml:tei:anchor
	\xmldoif{#1}{.[@type='indexbereich']}
		{\removeunwantedspaces}%
	\doifmode{FSBreferenzen}
		{%
		%% Falls der Anker im Sachapparat auftaucht, ..
		\doifmodeelse{FSBinSachapparat}
			%% .. wird er entsprechend exportiert.
			{\FSBexportelementlineapparatus{anchor}{\FSBattribute{xml:id}{\xmlatt{#1}{xml:id}}}{}%
			}
			%% Ansonsten wird die normale Routine aufgerufen..
			%% .. wenn er sich nicht im Lemma befindet.
			{\doifmodeelse{inLemma}{}%
				{\FSBexportelementline{anchor}{\FSBattribute{xml:id}{\xmlatt{#1}{xml:id}}}{}}%
			}
		}%
\stopxmlsetups

%% Bei Registereinträgen wird, wenn gewünscht, auch die Zeilennummern exportiert.
\startxmlsetups xml:tei:index
	\doifmode{FSBreferenzen}
		{%
		%% Falls der Eintrag im Sachapparat steht, ..
		\doifmodeelse{FSBinSachapparat}
			%% wird er entsprechend exportiert.
			{\FSBexportelementlineapparatus{index}
				{\FSBattribute{spanTo}{\xmlatt{#1}{spanTo}}\space
				\FSBattribute{indexName}{\xmlatt{#1}{indexName}}\space
				\FSBattribute{corresp}{\xmlatt{#1}{corresp}}\space
				\FSBattribute{term}{\xmltext{#1}{./term}}\space
				}
				{}%
			}
			%% Ansonsten wird dies ..
			%% .. wenn sich der Eintrag nicht im Lemma befindet ..
			{\doifmodeelse{inLemma}{}%
				{\FSBexportelementline{index}
					{\FSBattribute{spanTo}{\xmlatt{#1}{spanTo}}\space
					\FSBattribute{indexName}{\xmlatt{#1}{indexName}}\space
					\FSBattribute{corresp}{\xmlatt{#1}{corresp}}\space
					\FSBattribute{term}{\xmltext{#1}{./term}}\space
					%% .. anders vermerkt.
					\FSBattribute{location}{text}
					}
					{}}%
			}%
		}%
	\ignorespaces% 
	%\removeunwantedspaces%
%%%	\doifmode{register}
%%%		{\FSBortsregister[\xmltext{#1}{./term}]{\xmltext{#1}{./term}}}%
%%%	\startFSBortsregister[\xmlall{#1}{term}]{\xmlall{#1}{term}} *** TODO *****************************************
%%%	INDEX
%%%	\xmlflush{#1}
%%%	\FSBortsregister[\xmlall{#1}{term}]{TEST2}{TEST\xmlflush{#1}}
%%%	\stopFSBortsregister
\stopxmlsetups

\stopenvironment





\startenvironment env_register % Zum Einlesen des Registers
\project project_ediarum_context

%% ##################################################################
%% ##                       Basiselemente                          ##
%% ##################################################################

%% Generelle Registrierung des xml-Markups
\startxmlsetups xml:register:setups
	%% Dabei muss darauf geachtet werden, dass sich die Bestimmungen hier nicht mit dem eigentlichen Transformationsrichtlinien überschneiden.
	\xmlsetsetup{register}{*}{-}
	\xmlsetsetup{register}{register}{xml:register:root}
	\xmlsetsetup{register}{register/eintrag}{xml:register:eintrag}
\stopxmlsetups

%% Das Setup muss registriert werden
\xmlregisterdocumentsetup{register}{xml:register:setups}

%% ##################################################################
%% ##                          Elemente                            ##
%% ##################################################################

%% Die Wurzel wird weiterverarbeitet
\startxmlsetups xml:register:root
	%% Das Register wird zweispaltig in kleinerer Schrift gesetzt.
	\startcolumns[n=2,distance=10mm,tolerance=verytolerant,align=flushleft]
	\bgroup
	\switchtobodyfont[9pt]
	\xmlflush{#1}
	\egroup
	\stopcolumns
\stopxmlsetups

%% Jeder Eintrag wird ..
\startxmlsetups xml:register:eintrag
	%% .. mit dem Stichwort als Überschrift ..
	\FSBregistereintrag{\xmlcommand{#1}{./stichwort}{xml:register:eintrag:stichwort}}
	%% .. und Briefverweisen als Inhalt gesetzt.
	\xmlcommand{#1}{./brief}{xml:register:eintrag:brief}
	\par
	%% Nach dem Ende des Eintrages werden evtl. Subeinträge verarbeitet.
	\xmlcommand{#1}{./eintrag}{xml:register:eintrag:eintrag}
\stopxmlsetups

%% Der Eintragstitel wird gedruckt.
\startxmlsetups xml:register:eintrag:stichwort
	\xmlflush{#1}
\stopxmlsetups

%% Für jeden Brief ..
\startxmlsetups xml:register:eintrag:brief
	%% .. wird die Briefnummer gedruckt, und zwar kursiv, wenn nur Apparatverweise vorliegen.
	\xmldoifelse{#1}{./zeile[@stelle='text']}
		{\xmlatt{#1}{nummer}}
		{{\it\xmlatt{#1}{nummer}}}%
	%% Für die erste Textstelle wird die Zeile gedruckt ..
	\xmlcommand{#1}{./zeile[position()=1]}{xml:register:eintrag:brief:zeile}%
	%% .. und mit vorangehendem Punkt die weiteren Zeilennummern.
	\xmlcommand{#1}{./zeile[position()>1]}{xml:register:eintrag:brief:zeilepunkt}%
	\space
\stopxmlsetups

%% Wenn auf eine Stelle im Apparat verwiesen wird, wird der Zeilenbereich kursiv gedruckt, sonst normal.
\startxmlsetups xml:register:eintrag:brief:zeile
	\doifelse{\xmlatt{#1}{stelle}}{apparatus}
		%% In Einträgen in Apparaten steht im start und stop-Attribut jeweils der schon fertig formatierte Zeilenbereich, daher muss FSBlinesetup nicht aufgerufen werden.
		{\bgroup\switchtobodyfont[8pt]\low{\it\xmlatt{#1}{start}}\egroup}%
		%{\bgroup\switchtobodyfont[8pt]\low{\it\FSBlinesetup{1}{\xmlatt{#1}{start}}{1}{\xmlatt{#1}{stop}}}\egroup}%
		{\bgroup\switchtobodyfont[8pt]\low{\FSBlinesetup{1}{\xmlatt{#1}{start}}{1}{\xmlatt{#1}{stop}}}\egroup}%
\stopxmlsetups

%% Wenn auf eine Stelle im Apparat verwiesen wird, wird der Zeilenbereich mit vorangehendenem Punkt kursiv gedruckt, sonst normal.
\startxmlsetups xml:register:eintrag:brief:zeilepunkt
	\doifelse{\xmlatt{#1}{stelle}}{apparatus}
		%% In Einträgen in Apparaten steht im start und stop-Attribut jeweils der schon fertig formatierte Zeilenbereich, daher muss FSBlinesetup nicht aufgerufen werden.
		{\bgroup\switchtobodyfont[8pt]\low{.\it\xmlatt{#1}{start}}\egroup}%
		%\bgroup\switchtobodyfont[8pt]\low{.\it\FSBlinesetup{1}{\xmlatt{#1}{start}}{1}{\xmlatt{#1}{stop}}}\egroup}%
		{\bgroup\switchtobodyfont[8pt]\low{.\FSBlinesetup{1}{\xmlatt{#1}{start}}{1}{\xmlatt{#1}{stop}}}\egroup}%
\stopxmlsetups


%% Jeder Subeintrag wird ..
\startxmlsetups xml:register:eintrag:eintrag
	%% .. mit dem Stichwort als Überschrift ..
	\FSBregistersubeintrag{\xmlcommand{#1}{./stichwort}{xml:register:eintrag:stichwort}}
	%% .. und Briefverweisen als Inhalt gesetzt.
	\xmlcommand{#1}{./brief}{xml:register:eintrag:brief}
	\par
	%% Nach dem Ende des Subeintrages werden evtl. Subsubeinträge verarbeitet.
	\xmlcommand{#1}{./eintrag}{xml:register:eintrag:eintrag:eintrag}
\stopxmlsetups

%% Jeder Subsubeintrag wird ..
\startxmlsetups xml:register:eintrag:eintrag:eintrag
	%% .. mit dem Stichwort als Überschrift ..
	\FSBregistersubsubeintrag{\xmlcommand{#1}{./stichwort}{xml:register:eintrag:stichwort}}
	%% .. und Briefverweisen als Inhalt gesetzt.
	\xmlcommand{#1}{./brief}{xml:register:eintrag:brief}
	\par
\stopxmlsetups

\stopenvironment

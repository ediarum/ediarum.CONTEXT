\startenvironment env_TEI2tex % Die TEI nach TeX Konvertierung
\project project_ediarum_context

%% ##################################################################
%% ##                      Basiselemente                           ##
%% ##################################################################

%% Generelle Registrierung des xml-Markups
\setupxml[default=none]
\startxmlsetups xml:tei:setups
	%% Alle nicht explizit definierten Elemente bleiben unberücksichtigt.
	\xmlsetsetup{tei}{*}{-}
	%% Der Startpunkt wird registriert.
	\xmlsetsetup{tei}{TEI}{xml:tei:*}
	%% Die Textstrukturelemente folgen.
	\xmlsetsetup{tei}{text|body|div|postscript}{xml:tei:*}
	%% Die Absatzelemente sind:
	\xmlsetsetup{tei}{opener|closer|p|list|lg}{xml:tei:*}
	%% Die Elemente aus opener und closer sind diese:
	\xmlsetsetup{tei}{address|dateline|salute|signed}{xml:tei:*}%TODO Sachkommentare
	%% Nun folgen die einfachen Textelemente.
	\xmlsetsetup{tei}{bibl|damage|date|ex|foreign|gap|hi|lb|persName|placeName|pb|unclear}{xml:tei:*}%TODO Sachkommentare
	%% Neue Textelemente für ediarum.basis
	\xmlsetsetup{tei}{space|milestone|g}{xml:tei:*}
	%% Workaround für das #-Zeichen - auf diese Weise wird es erkannt
	%% ...Für hi
	\xmlsetsetup{tei}{hi[@rendition='\#u']}{xml:tei:hi:u}
	\xmlsetsetup{tei}{hi[@rendition='\#uu']}{xml:tei:hi:uu}
	\xmlsetsetup{tei}{hi[@rendition='\#uuu']}{xml:tei:hi:uuu}
	\xmlsetsetup{tei}{hi[@rendition='\#sup']}{xml:tei:hi:sup}
	\xmlsetsetup{tei}{hi[@rendition='\#sub']}{xml:tei:hi:sub}
	%% ...Horizontale Linie
	\xmlsetsetup{tei}{milestone[@rendition='\#hr']}{xml:tei:milestone:hr} % NOT WORKING YET	
	%% ...Sonderzeichen
	\xmlsetsetup{tei}{g[@ref='\#typoHyphen']}{xml:tei:g:typoHyphen}
	%% Nun folgen die Tabellenelemente.
	\xmlsetsetup{tei}{table|row|cell|tr}{xml:tei:*}
	%% Als nächstes kommen alle Elemente mit Apparateintrag.
	\xmlsetsetup{tei}{add|choice|del|note|seg|subst|supplied|surplus}{xml:tei:*}%TODO Sachkommentare
	%% Workaround für das #-Zeichen - auf diese Weise wird es erkannt
	%% ...Für delete
	\xmlsetsetup{tei}{del[@rendition='\#s']}{xml:tei:del:s}
	\xmlsetsetup{tei}{del[@rendition='\#ow']}{xml:tei:del:ow}
	\xmlsetsetup{tei}{del[@rendition='\#erased']}{xml:tei:del:erased}

	%% Dieses sind die Elemente für Register und Verweise
	\xmlsetsetup{tei}{anchor|index|ref|rs|persname}{xml:tei:*}%TODO Sachkommentare hinzu Althof: rs - persname

\stopxmlsetups

%% Das Register wird unter 'tei' registriert.
\xmlregisterdocumentsetup{tei}{xml:tei:setups}



%% ##################################################################
%% ##                     Listen						           ##
%% ##################################################################

%% Listen definieren % funktioniert noch nicht...
\setupitemize
	[1]
	[margin=no,
    distance=1em,
    stopper=,
    left=- ,
    right=:]


%% ##################################################################
%% ##               Der Einstieg in das Dokument                   ##
%% ##################################################################

%% Der Brief wird generell vorbereitet und die Metadaten werden ausgelesen.
\startxmlsetups xml:tei:TEI
	%% Falls die Verarbeitung von externen Referenzen stattfinden soll, werden die Briefdaten extern gespeichert, ..
	\doifmode{FSBreferenzen}
		{\FSBfwriteT%
		%% .. und zwar mit Briefnummer ..
		\xmldoifelsetext{#1}{./teiHeader/fileDesc/titleStmt/title/idno}
			{\FSBfwrite{<file xml:id="\xmlatt{#1}{xml:id}" type="Brief"}%
			\FSBfwrite{ idno="\xmltext{#1}{./teiHeader/fileDesc/titleStmt/title/idno}">}%
			}
			%% .. oder ohne.
			{\FSBfwrite{<file xml:id="\xmlatt{#1}{xml:id}" type="Brief"}%
			\FSBfwrite{ idno="\xmlatt{#1}{xml:id}">}%
			}%
		\FSBfwriteN%
		}
	%% Falls ein Deckblatt für einen einzelnen Brief erzeugt werden soll, werden dort die Metadaten hingeschrieben.
	\doifmode{deckblatt}
		{\xmlcommand{#1}{.}{xml:tei:deckblatt}}%
	%% Hier beginnt die Verarbeitung für den Druck, ..
	%% .. dabei wird entschieden, ob das Regest erscheint oder der Editionstext.
	\xmldoifelse{#1}{div[@type='nurRegest']}
		{% Falls nur das Regest erscheinen soll, ..
		%% .. werden nur die Briefnummer und der Titel ..
		\bgroup
		\switchtobodyfont[12pt]
		\it
		%% .. als Markierung für die Kopfzeile ..
		\xmldoifelse{#1}{./teiHeader/fileDesc/titleStmt/title/idno}
			{\marking[FSBbriefnummer]{\xmltext{#1}{./teiHeader/fileDesc/titleStmt/title/idno}}
			}
			{\marking[FSBbriefnummer]{\xmlatt{#1}{xml:id}}
			}
		\marking[FSBbriefdatum]{\xmltext{#1}{./teiHeader/profileDesc/creation/date[@type='erstellung']}}
		%% .. und als Überschrift verwendet.
		\xmldoifelse{#1}{./teiHeader/fileDesc/titleStmt/title/idno}
			{*\xmltext{#1}{./teiHeader/fileDesc/titleStmt/title/idno}.\hskip2mm
			}
			{*\xmlatt{#1}{xml:id}.\hskip2mm
			}
		\xmltext{#1}{./teiHeader/fileDesc/titleStmt/title}
		\egroup
		%% Dann erscheint das Regest im Text ..
		\blank[big]
		\bgroup\it\xmltext{#1}{./teiHeader/fileDesc/sourceDesc/msDesc/msContents/summary}\egroup
		%% .. und schließlich noch die zugehörige Bemerkung in der Fußnote.
		\FSBdotextapparat{%
			\bgroup\bi *\xmltext{#1}{./teiHeader/fileDesc/titleStmt/title/idno}.\egroup
			\hskip5mm
			\bgroup\it\xmltext{#1}{./teiHeader/fileDesc/sourceDesc/msDesc/msContents/p/note}\egroup
			}
		\blank[2*big]
		}
		{% Im Normalfall wird der Brief mit Überlieferung und Editionstext gedruckt.
		%% Auch dann werden die Briefnummer und der Titel ..
		\bgroup
		\switchtobodyfont[12pt]
		\it
		%% .. als Markierung für die Kopfzeile .. (geändert Althof: /idno gelöscht)
		\xmldoifelse{#1}{./teiHeader/fileDesc/titleStmt/title}
			{
				%\marking[FSBbriefnummer]{\xmltext{#1}{./teiHeader/fileDesc/titleStmt/title}}
				%% ...in FSBnotebooktitle hinterlegt (für Gödel-Notizbücher)
				\marking[FSBnotebooktitle]{\xmltext{#1}{./teiHeader/fileDesc/titleStmt/title}}
			}
			{
				%\marking[FSBbriefnummer]{\xmlatt{#1}{xml:id}}
				\marking[FSBnotebooktitle]{\xmlatt{#1}{xml:id}}
			}
		\marking[FSBbriefdatum]{\xmltext{#1}{./teiHeader/profileDesc/creation/date[@type='erstellung']}}
		%% .. und als Überschrift verwendet. (geändert Althof: keine Überschrift)
		%\xmldoifelse{#1}{./teiHeader/fileDesc/titleStmt/title}
		%	{
		%	\tfb % größere Schrift
		%	\xmltext{#1}{./teiHeader/fileDesc/titleStmt/title}.\hskip2mm
		%	}
		%	{\xmlatt{#1}{xml:id}.\hskip2mm
		%	}
		%\xmltext{#1}{./teiHeader/fileDesc/titleStmt/title}
		\egroup
		%% Im Textapparat erscheint dann ..
		%\FSBdotextapparat{%
			%% .. zu der Briefnummer ..
		%	\bgroup\bi \xmltext{#1}{./teiHeader/fileDesc/titleStmt/title/idno}.\egroup
			%% .. die Überlieferung, ..
		%	\hskip5mm
		%	\bgroup
		%		\it Überlieferung:
		%		\xmlcommand{#1}{./teiHeader/fileDesc/sourceDesc}{xml:tei:ueberlieferung}
		%	\egroup
			%% .. und eventuell die Adressierung.
		%	\xmldoif{#1}{./text/body/opener/address/addrLine}
		%		{\xmldoifelse{#1}{./text/body/opener/address/addrLine/add[@place]}
		%			{\bgroup\it; Adresse \xmlattribute{#1}{./text/body/opener/address/addrLine/add}{place}:\egroup\ \xmltext{#1}{./text/body/opener/address/addrLine/add}}
		%			{\bgroup\it; Adresse:\egroup\ \xmltext{#1}{./text/body/opener/address/addrLine}}
		%		}
		%	}
		\blank[big]
		%% Schließlich wird der Editionstext selbst gedruckt.
		\xmlall{#1}{./text}
		\blank[2*big]
		}%
	%% Falls die Referenzen exportiert wurden, wird das entsprechend abgeschlossen.
	\doifmode{FSBreferenzen}{\FSBfwriteT\FSBfwrite{</file>}\FSBfwriteN}
\stopxmlsetups

%% ##################################################################
%% ##               Elemente für das Deckblatt                     ##
%% ##################################################################

%% Die Metadaten werden in das Deckblatt geschrieben.
\startxmlsetups xml:tei:deckblatt
	%% Es wird eine Seite außerhalb der Seitenzählung und des normalen Setups wird erzeugt.
	\startstandardmakeup
	%% Zunächst kommt der Titel ..
	\bgroup\bfa \xmltext{#1}{./teiHeader/fileDesc/titleStmt/title}\egroup
	%% .. und dann die weiteren Informationen.
	\blank[small]
	\bgroup
	%\xmltext{#1}{./teiHeader/fileDesc/editionStmt/edition/title[@type='editionstitel']}\crlf
	%\xmltext{#1}{./teiHeader/fileDesc/editionStmt/edition/title[@type='abteilungstitel']}\crlf
	\xmltext{#1}{./teiHeader/fileDesc/editionStmt/edition/title}\crlf
	\egroup
	%Briefnr. \xmltext{#1}{./teiHeader/fileDesc/titleStmt/title/idno}
	\blank[3*big]
	\xmlcommand{#1}{./teiHeader/fileDesc/editionStmt/respStmt}{xml:tei:deckblatt:respStmt}
	\crlf
	\bgroup\it herausgegeben im Auftrag von\egroup
	\crlf
	\xmltext{#1}{./teiHeader/fileDesc/publicationStmt/publisher/ref} %% geändert Althof (hinzu: /ref)
	%% Zum Schluß folgt noch die Überlieferung ..
	\blank
	%% .. für alle Manuskripte ..
	\xmlcommand{#1}{./teiHeader/fileDesc/sourceDesc/msDesc}{xml:tei:deckblatt:msDesc}
	%% .. und für alle Textzeugen ..
	\xmlcommand{#1}{./teiHeader/fileDesc/sourceDesc/listWit/witness}{xml:tei:deckblatt:witness}
	%% .. und das Regest.
	\blank[3*big]
	\bgroup\it Zusammenfassung\egroup\crlf
	\xmltext{#1}{./teiHeader/fileDesc/sourceDesc/msDesc/msContents/summary}
	\blank
	\bgroup\it Bemerkung\egroup\crlf
	\xmltext{#1}{./teiHeader/fileDesc/sourceDesc/msDesc/msContents/summary/note}
	\stopstandardmakeup
\stopxmlsetups

%% Die Informationen zu den Drucken werden für das Deckblatt ausgegeben.
\startxmlsetups xml:tei:deckblatt:bibl
	\bgroup\it\Word{\xmlatt{#1}{n}}\egroup\crlf
	\xmltext{#1}{.}
	\blank
\stopxmlsetups

%% Die Informationen zum Manuskript werden hier für das Deckblatt ausgegeben.
\startxmlsetups xml:tei:deckblatt:msDesc
	\bgroup\it \Word{\xmlatt{#1}{n}}\egroup\crlf
	\xmltext{#1}{./msIdentifier/institution}\crlf
	\xmltext{#1}{./msIdentifier/collection}\crlf % geändert Althof: repository in collection
	\xmltext{#1}{./msIdentifier/idno}
	\blank
\stopxmlsetups

%% Für das Deckblatt werden die verantwortlichen Personen mit ihrer Rolle genannt.
\startxmlsetups xml:tei:deckblatt:respStmt
	\xmltext{#1}{./name} (\bgroup\it\xmltext{#1}{./resp}\egroup)\crlf
\stopxmlsetups

%% Die Informationen zu den Textzeugen werden hier für das Deckblatt ausgegeben
\startxmlsetups xml:tei:deckblatt:witness
	\xmlcommand{#1}{./msDesc}{xml:tei:deckblatt:msDesc}
	\xmlcommand{#1}{./bibl}{xml:tei:deckblatt:bibl}
\stopxmlsetups

%% ##################################################################
%% ##              Element für die Überlieferung                   ##
%% ##################################################################

%% Die Überlieferung wird im Textapparat behandelt.
%\startxmlsetups xml:tei:ueberlieferung
	%% Zunächst wird der Nachweis zu einer evtl. Handschrift ausgegeben ..
%	\xmldoif{#1}{msDesc[@rend='Handschrift']}
%		{\ H: \xmltext{#1}{msDesc[@rend='Handschrift']/msIdentifier/idno}}
	%% .. mit Trennzeichen, falls es Handschrift und Abschrift gibt, ..
%	\doif{\xmlcount{#1}{msDesc}}{2}
%		{;}
	%% .. dann kommt eine evtl. Abschrift.
%	\xmldoif{#1}{msDesc[@rend='Abschrift']}
%		{\ h: \xmltext{#1}{msDesc[@rend='Abschrift']/msIdentifier/idno}}
	%% Ein weiterer Trenner folgt, wenn es Manuskripte und Drucke gibt.
%	\xmldoif{#1}{msDesc}{\xmldoif{#1}{bibl}
%		{;}}%
	%% Falls es nur einen Druck gibt ..
%	\doif{\xmlcount{#1}{bibl}}{1}
%		{% .. wird er mit D eingeleitet, ..
%		\ D: \xmltext{#1}{bibl}
%		}%
	%% bei mehreren Drucken ..
%	\doif{\xmlcount{#1}{bibl}}{2}
%		{% werden diese nummeriert.
%		\ D1: \xmlindex{#1}{bibl}{1};
%		\ D2: \xmlindex{#1}{bibl}{2}
%		}%
	%% Falls es eine Abschrift und Drucke gibt und die Abschrift Textgrundlage ist, wird dies gedruckt.
%	\xmldoif{#1}{./msDesc[@n='Abschrift']}{\xmldoif{#1}{bibl}
%		{; Textgrundlage: h}}
	%% Wenn der erste Druck Textgrundlage ist, ..
%	\xmldoif{#1}{./bibl[@n='1']}
%		{% .. und falls es eine Abschrift und einen Druck gibt, reicht ein D, ..
%		\xmldoif{#1}{msDesc[@rend='Abschrift']}{\doif{\xmlcount{#1}{bibl}}{1}
%			{; Textgrundlage: D}}
		%% .. bei mehreren Drucken dagegen, braucht es ein D1.
%		\doif{\xmlcount{#1}{bibl}}{2}
%			{; Textgrundlage: D1}
%		}%
	%% Wenn der zweite Druck Textgrundlage ist, gibt es ein D2.
%	\xmldoif{#1}{./bibl[@n='2']}
%		{; Textgrundlage: D2}
%\stopxmlsetups

%% ##################################################################
%% ##              Die Strukturelemente des Textes                 ##
%% ##################################################################

%% Der Einstieg in die Ausgabe des edierten Textes
\startxmlsetups xml:tei:text
	%% Zuerst wird die Zeilennummerierung auf null gesetzt, ..
	\FSBresetlinenumbering[FSBdefaultlinenumbering]
	%%  .. dann wird der Inhalt ausgegeben, ..
	\xmlall{#1}{./body}
	%% .. und zum Schluss die vorhandenen Autorenfussnoten gedruckt.
	%\placenotes[FSBautornote] %% geändert Althof - mit Kommentar versehen
\stopxmlsetups

%% Im Textkörper tauchen nur div und postscript auf.
\startxmlsetups xml:tei:body
	\xmlall{#1}{./(div|postscript)} 
\stopxmlsetups

%% Jeder Textbereich kann opener, p, list, lg und closer enthalten.
\startxmlsetups xml:tei:div
	\xmlall{#1}{./(opener|p|list|lg|closer)}
\stopxmlsetups

%% Ein Postskriptum wird wie ein div behandelt.
\startxmlsetups xml:tei:postscript
	\xmlall{#1}{./(opener|p|list|lg|closer)}
\stopxmlsetups

%% ##################################################################
%% ##                   Die Absatzelemente                         ##
%% ##################################################################

%% Im Briefkopf stehen Adresse und Datum.
\startxmlsetups xml:tei:opener
	%% Zuerst wird sichergestellt, daß wir in einem neuen Absatz sind, ..
	\par
	%% .. dann beginnt die Zeilennummerierung.
	\startlinenumbering[FSBdefaultlinenumbering][continue]
	%% Die Adresse und das Datum stehen rechts.
	\startalignment[flushright]
	\xmlall{#1}{./(address|dateline)}
	\stopalignment
	\stoplinenumbering
	%% Nach einem kleinen Durchschuss ..
	\blank[small]
	%% .. wird linksbündig die Begrüßung eingefügt.
	\startlinenumbering[FSBdefaultlinenumbering][continue]
	\xmlall{#1}{./salute}
	\stoplinenumbering
	%% Nach dem Briefkopf steht der erste Absatz nicht eingerückt, aber die folgenden.
	\indenting[yes,next]
\stopxmlsetups

%% Am Briefschluss steht alles linksbündig.
\startxmlsetups xml:tei:closer
	%% Die Zeilennummerierung wird fortgeführt, ..
	\startlinenumbering[FSBdefaultlinenumbering][continue]
	%% .. dann folgen Ort, Datierung, Abschied und Unterschrift.
	\xmlall{#1}{./(address|dateline|salute|signed)}
	\stoplinenumbering
\stopxmlsetups

% Die normalen Absätze enthalten den meisten Text.
\startxmlsetups xml:tei:p
	%% Zuerst beginnt der neue Paragraph
	\par
	%% ...mit einem Abstand zum vorangehenden
	%\blank[small]
	%\hskip5mm
	\vskip5mm
	%% Falls der Absatz zentriert erscheinen soll, ..
	\doifelse{\xmlatt{#1}{rend}}{align(center)}
		{% .. wird er entsprechend gesetzt ..
		\bgroup
		\startalignment[middle]
		%% .. mit Zeilennummerierung, ..
		\startlinenumbering[FSBdefaultlinenumbering][continue]
		\xmlflush{#1}%TODO Sachkommentare
		%\xmlall{#1}{bibl|damage|date|ex|foreign|gap|hi|lb|persName|placeName|pb|unclear|add|choice|del|note|seg|subst|supplied|surplus|anchor|index|ref}%TODO Sachkommentare
		\stoplinenumbering
		\stopalignment
		\egroup
		}
		{% .. ansonsten bleibt er auch mit Zeilennummerierung linksbündig.
		\startlinenumbering[FSBdefaultlinenumbering][continue]
		\xmlstripnolines{#1}{.}
		\xmlflush{#1}%TODO Sachkommentare
		%\xmlall{#1}{bibl|damage|date|ex|foreign|gap|hi|lb|persName|placeName|pb|unclear|add|choice|del|note|seg|subst|supplied|surplus|anchor|index|ref}%TODO Sachkommentare
		\stoplinenumbering
		}%
\stopxmlsetups

%% Eine Liste mit verschiedenen Einträgen ..
\startxmlsetups xml:tei:list
	%% .. beginnt auch in einem neuen Absatz ..
	\par
	%% .. und natürlich der Zeilennummerierung.
	\startlinenumbering[FSBdefaultlinenumbering][continue]
	%% Falls es eine Liste mit Aufzählungspunkten sein soll, ..
	 \doif{\xmlatt{#1}{type}}{bulleted}
		{% .. werden diese einzelnen Elemente damit aufgerufen, ..
			\startitemize[1]
			\xmlcommand{#1}{./item}{xml:tei:list:item}
			\xmlflush{#1}
			\stopitemize%
		}
	%% .. oder das gleiche bei einer nummerierten Liste.
	\doifelse{\xmlatt{#1}{type}}{ordered}
	{
		\startitemize[n,packed]
		\xmlcommand{#1}{./item}{xml:tei:list:item}
		\stopitemize%
	}
	% oder eine Liste ohne Attribut (hinzu Althof)
	%\doifempty{\xmlatt{#1}{type}} % TODO eigene Bedingung für Liste ohne Attribut (so nicht funktionsfähig)
	%\doif{\xmlatt{#1}{type}}{}{ % TODO eigene Bedingung für Liste ohne Attribut (so nicht funktionsfähig)
	{
		\startitemize[2]
		\xmlcommand{#1}{./item}{xml:tei:list:item}
		\xmlflush{#1}
		\stopitemize%
	}
	\stoplinenumbering
\stopxmlsetups

%% Die einzelnen Elemente einer Liste werden entsprechend gekennzeichnet.
\startxmlsetups xml:tei:list:item
	\item \xmlflush{#1}
\stopxmlsetups

%% Eine Reihe von einzelnen Zeilen werden in einer Zeilengruppe behandelt.
\startxmlsetups xml:tei:lg
	%% Zunächst fängt ein neuer Absatz an ..
	\par
	%% .. mit einem mittleren Durchschuss ..
	\blank[medium]
	%% .. und der Zeilennummerierung, ..
	\startlinenumbering[FSBdefaultlinenumbering][continue]
	%% .. dann kommen die einzelnen Zeilen.
	\xmlcommand{#1}{./l}{xml:tei:lg:l}
	\stoplinenumbering
	\blank
	\noindentation
\stopxmlsetups

%% Die einzelnen Zeilen stehen jeweils in einer eigenen Zeile.
\startxmlsetups xml:tei:lg:l
	\par\xmlflush{#1}
\stopxmlsetups

%% ##################################################################
%% ##        Die Elemente aus Briefkopf und Briefschluss           ##
%% ##################################################################

%% Die Adresse wird hingeschrieben, ..
\startxmlsetups xml:tei:address
	%% .. da evtl. Zeilenumbrüche hier anders gehandhabt werden als normal, gibt es den Modus FSBlinebreak
	\enablemode[FSBlinebreak]
	\xmlcommand{#1}{./addrLine}{xml:tei:address:addrLine}
	\xmlflush{#1}
	\disablemode[FSBlinebreak]
\stopxmlsetups

%% Nach einem kleinen Durchschuss folgt die Adresse.
\startxmlsetups xml:tei:address:addrLine
	\par\blank[small]\xmlflush{#1}
\stopxmlsetups

%% Nach einem kleinen Durchschuss folgt das Datum.
\startxmlsetups xml:tei:dateline
	\par\blank[small]\xmlflush{#1}
\stopxmlsetups

%% Nach einem Absatz folgt die Begrüßung.
\startxmlsetups xml:tei:salute
	\par\xmlflush{#1}
\stopxmlsetups

%% Die Unterschrift ist in einem eigenen Absatz rechtsbündig.
\startxmlsetups xml:tei:signed
	\par\rightaligned{\xmlflush{#1}}
\stopxmlsetups

%% ##################################################################
%% ##                Die einfachen Textelemente                    ##
%% ##################################################################

%% Gekennzeichnete Werke werden nur ausgegeben.
\startxmlsetups xml:tei:bibl
	\xmlstrip{#1}{.}
	\xmlflush{#1}
\stopxmlsetups

%% Beschädigte Stellen mit kursiven eckigen Klammern gekennzeichnet.
\startxmlsetups xml:tei:damage
	\bgroup\it [\space]\egroup
\stopxmlsetups

%% Datumsangaben werden nur ausgegeben.
\startxmlsetups xml:tei:date
	\xmlstrip{#1}{.}
	\xmlflush{#1}
\stopxmlsetups

%% Ergänzungen des Herausgebers werden kursiv gesetzt.
\startxmlsetups xml:tei:ex
	\bgroup\it\ignorespaces\xmlflush{#1}\removeunwantedspaces\egroup
\stopxmlsetups

%% Wenn ein Abschnitt in einer anderen Sprache als deutsch vorliegt.
\startxmlsetups xml:tei:foreign
	%% Falls es sich um einen griechischen Text handelt, ..
	\doifelse{\xmlatt{#1}{xml:lang}}{grc}
		{% .. wird die Schrift auf eine griechische Zeichen unterstützende umgestellt.
		{\FSBgreek{\xmlflush{#1}}}
		}
		{% Bei allen anderen Sprachen gibt es keine Veränderung.
		\xmlflush{#1}}
\stopxmlsetups

%% Eine Lücke im Manuskript wird durch Winkel angedeutet.
\startxmlsetups xml:tei:gap
	\FSBlfloor\ \FSBrceil
\stopxmlsetups

%% Sonderanweisungen für hi....
\startxmlsetups xml:tei:hi:u
	%% für einfache Unterstreichung
		\underbar{\ignorespaces \xmlflush{#1}} % mit \removeunwantedspaces werden auch Leerzeichen zwischen Worten gelöscht...
\stopxmlsetups

\startxmlsetups xml:tei:hi:uu
	%% für doppelte Unterstreichung
	\underbar{\underbar{\xmlflush{#1}}}%
\stopxmlsetups

\startxmlsetups xml:tei:hi:uuu
	%% für dreifache Unterstreichung
	\underbar{\underbar{\underbar{\xmlflush{#1}}}}%
\stopxmlsetups

\startxmlsetups xml:tei:hi:sup
	%% hochgestellt
	\high{\xmlflush{#1}}%
\stopxmlsetups

\startxmlsetups xml:tei:hi:sub
	%% tiefgestellt
	\low{\xmlflush{#1}}%
\stopxmlsetups

\startxmlsetups xml:tei:hi
	%% sonst einfach alle anderen hi (ohne Attribut) gesperrt
		\xmlstrip{#1}{.}
		\FSBgesperrt{\xmlflush{#1}}%
\stopxmlsetups

%% Sonderanweisungen für Horizontale Linie
\startxmlsetups xml:tei:milestone:hr
	%% horizontale Linie
	\crlf
	line
	\crlf
\stopxmlsetups

%% Zeilenwechsel im Manuskript werden nur an Stellen berücksichtigt, an denen dies explizit gewünscht ist (etwa bei der Anschrift).
\startxmlsetups xml:tei:lb
	\doifmodeelse{FSBlinebreak}
		{\crlf}
		{}%
\stopxmlsetups

%% Sonderanweisung für Zeilenwechsel mit Trennstrich (nur der Trennstrich wird gesetzt)
\startxmlsetups xml:tei:g:typoHyphen
	-
\stopxmlsetups

%% Personen werden nur ausgegeben.
\startxmlsetups xml:tei:persName
	\xmlstrip{#1}{.}
	\xmlflush{#1}
\stopxmlsetups

%% Auch Orte werden nur ausgegeben.
\startxmlsetups xml:tei:placeName
	\xmlstrip{#1}{.}
	\xmlflush{#1}
\stopxmlsetups

%% Auch Verweise ins Sachregister werden nur ausgegeben. % hinzu Althof
\startxmlsetups xml:tei:rs
	\xmlstrip{#1}{.}
	\xmlflush{#1}
\stopxmlsetups

%% Ein Seitenwechsel im Manuskript wird im Druck durch einen senkrechten Strich dargestellt.
\startxmlsetups xml:tei:pb
	|(\xmlatt{#1}{n})
\stopxmlsetups

%% Unklare Stellen werden in Winkeln dargestellt.
\startxmlsetups xml:tei:unclear
	\FSBlfloor\xmlflush{#1}\FSBrceil
\stopxmlsetups

%% Spatium (wie bei gap mit Winkeln dargestellt)
\startxmlsetups xml:tei:space
	 \FSBlfloor\ \FSBrceil
\stopxmlsetups

%% Tabellen... (hinzu Althof)
\startxmlsetups xml:tei:table
%% ...mit 3 Spalten
	\doif	{\xmlatt{#1}{cols}}{3}
	{
		\vskip2mm
		\setupTABLE[c][1,2,3][align=right,width=.33\textwidth]
		\bTABLE[frame=on,split=yes]
		\xmlflush{#1}
		\eTABLE
		\vskip2mm
	}
	%% ...mit 2 Spalten
	\doif	{\xmlatt{#1}{cols}}{2}
	{
		\vskip2mm
		\setupTABLE[c][1,2][align=right,width=.5\textwidth]
		\bTABLE[frame=on,split=yes]
		\xmlflush{#1}
		\eTABLE
		\vskip2mm
	}
	%% ...mit 1 Spalte
	\doif	{\xmlatt{#1}{cols}}{1} {
		\vskip2mm
		\setupTABLE[c][1][align=right,width=1\textwidth]
		\bTABLE[frame=on,split=yes]
		\xmlflush{#1}
		\eTABLE
		\vskip2mm
	}
	%% ...falls kein Attribut zur Spaltenzahl angegeben ist, ist die Tabelle nur 1 Spalte breit
	%\doif{\expanded{\xmlatt{#1}{cols}}}{}{ %% TODO Bedingung überprüfen
	%\xmldoifelseempty	{#1}{./attribute(cols)} { %% TODO Bedingung überprüfen
	\doif{\xmlatt{#1}{cols}}{}{
	 \vskip2mm
		\setupTABLE[c][1][align=right,width=1\textwidth]
		\bTABLE[frame=on,split=yes]
		\xmlflush{#1}
		\eTABLE
		\vskip2mm
	}
\stopxmlsetups

\startxmlsetups	xml:tei:row
	\bTR \xmlflush{#1}\eTR
\stopxmlsetups

\startxmlsetups xml:tei:th
	\bTD[align=middle,style=bold]
	\xmlflush{#1}\eTD
\stopxmlsetups

\startxmlsetups xml:tei:cell
	\bTD \xmlflush{#1}\eTD
\stopxmlsetups


%% ##################################################################
%% ##             Alle Elemente mit Apparateintrag                 ##
%% ##################################################################

%% Eine einfache Einfügung im Text wird mit dem Ort in die Fußnote geschrieben.
\startxmlsetups xml:tei:add
	\FSBtextapparat{\xmlflush{#1}}
		{\xmlflush{#1}}{\xmlatt{#1}{place}}{}
\stopxmlsetups

%% Eine Textanmerkung vom Herausgeber kann ..
\startxmlsetups xml:tei:choice
	%% .. entweder eine Korrektur ..
	\xmldoif{#1}{./corr}
		{% .. (mit niedriger Wahrscheinlichkeit nur als Konjektur im Apparat ..
		\doif{\xmlattribute{#1}{./corr}{cert}}{low}
			{\FSBtextapparat{\xmlcommand{#1}{./sic}{xml:tei:choice:sic}}
				{\xmlcommand{#1}{./sic}{xml:tei:choice:sic}}{Kj.}{\xmlcommand{#1}{./corr}{xml:tei:choice:corr}}
			}%
		%% .. oder mit hoher Wahrscheinlichkeit als Emendation im Text mit Anmerkung) ..
		\doif{\xmlattribute{#1}{./corr}{cert}}{high}
			{\FSBtextapparat{\xmlcommand{#1}{./corr}{xml:tei:choice:corr}}
				{\xmlcommand{#1}{./corr}{xml:tei:choice:corr}}{}{\xmlcommand{#1}{./sic}{xml:tei:choice:sic}}
			}%
		}%
	%% .. die Auflösung einer Abkürzung ..
	\xmldoif{#1}{./abbr}
		{% .. im Textapparat ..
		\FSBtextapparat{\xmlcommand{#1}{./abbr}{xml:tei:choice:abbr}}
			{\xmlcommand{#1}{./abbr}{xml:tei:choice:abbr}}{Abk. wohl für}{\xmlcommand{#1}{./expan}{xml:tei:choice:expan}}
		}%
	%% .. oder den Hinweis auf eine weitere Lesart beinhalten, ..
	\xmldoif{#1}{./unclear}
		{% .. der dann auch im Textapparat vermerkt wird.
		\FSBtextapparat{\xmlcommand{#1}{./unclear[1]}{xml:tei:choice:unclear}}
			{\xmlcommand{#1}{./unclear[1]}{xml:tei:choice:unclear}}{oder}{\xmlcommand{#1}{./unclear[2]}{xml:tei:choice:unclear}}
		}%
\stopxmlsetups

%% Eine Abkürzung steht im choice zusammen mit der Auflösung.
\startxmlsetups xml:tei:choice:abbr
	\xmlflush{#1}
\stopxmlsetups

%% Eine Korrektur steht mit dem Manuskripttext im choice
\startxmlsetups xml:tei:choice:corr
	\xmlflush{#1}
\stopxmlsetups

%% Die Auflösung einer Abkkürzung steht wird nur ausgegeben.
\startxmlsetups xml:tei:choice:expan
	\xmlflush{#1}
\stopxmlsetups

%% Der Manuskripttext zu einer Herausgeberkorrektur steht im choice.
\startxmlsetups xml:tei:choice:sic
	\xmlflush{#1}
\stopxmlsetups

%% Ein unklarer Ausdruck im choice wird auch in Winkel gesetzt.
\startxmlsetups xml:tei:choice:unclear
	\FSBlfloor\xmlflush{#1}\FSBrceil
\stopxmlsetups

%% Einfache Löschung für Ediarum (hinzu Althof)
\startxmlsetups xml:tei:del
	\FSBtextapparat{||} % Löschungszeichen für den Text
	{\type{||}}{\xmlflush{#1}}{gelöscht} % {Löschungszeichen als Lemma}{gelöschter Text}{Anmerkung}
\stopxmlsetups

%% Einfache Löschung für Ediarum (hinzu Althof)
\startxmlsetups xml:tei:del:s
	\FSBtextapparat{||} % Löschungszeichen für den Text
	{\type{||}}{\xmlflush{#1}}{durchgestrichen vom Autor} % {Löschungszeichen als Lemma}{gelöschter Text}{Anmerkung}
\stopxmlsetups

%% Einfache Löschung für Ediarum (hinzu Althof)
\startxmlsetups xml:tei:del:ow
	\FSBtextapparat{||} % Löschungszeichen für den Text
	{\type{||}}{\xmlflush{#1}}{überschrieben vom Autor} % {Löschungszeichen als Lemma}{gelöschter Text}{Anmerkung}
\stopxmlsetups

%% Einfache Löschung für Ediarum (hinzu Althof)
\startxmlsetups xml:tei:del:erased
	\FSBtextapparat{||} % Löschungszeichen für den Text
	{\type{||}}{\xmlflush{#1}}{radiert} % {Löschungszeichen als Lemma}{gelöschter Text}{Anmerkung}
\stopxmlsetups

%% Eine einfache Löschung ..
%\startxmlsetups xml:tei:del
	%% .. wird mit dem vorangehenden Wort ..
%	\xmldoif{#1}{.[@prev]}
%		{\FSBtextapparat{}
%			{\FSBwikitotex{\xmlatt{#1}{prev}}}{folgt}{\FSBlang\ignorespaces\xmlflush{#1}\removeunwantedspaces\FSBrang}
%		}%
	%% .. oder mit dem folgenden als Lemma im Textapparat in spitze Klammern gesetzt.
%	\xmldoif{#1}{.[@next]}
%		{\FSBtextapparat{}
%			{\FSBwikitotex{\xmlatt{#1}{next}}}{davor}{\FSBlang\ignorespaces\xmlflush{#1}\removeunwantedspaces\FSBrang}
%		}%
%\stopxmlsetups

%% Eine Autorenfußnote wird als Fußnote im Text gedruckt. (hinzu Althof)
\startxmlsetups xml:tei:note
	%\FSBautornote{{\it\xmlatt{#1}{place}:} \xmlflush{#1}} %% urpsprüngliche Version - geändert Althof 
	\doif{\xmlatt{#1}{place}}{mBottom}
		{
			%\FSBautornote{{\it\xmlatt{#1}{place}:} \xmlflush{#1}} %% urpsprüngliche Version - geändert Althof 
			\FSBfootnote{Anmerkung unten: \xmlflush{#1}}
		}
	\doif{\xmlatt{#1}{place}}{right}
		{
			\FSBfootnote{Anmerkung rechts: \xmlflush{#1}}
		}
		\doif{\xmlatt{#1}{place}}{left}
		{
			\FSBfootnote{Anmerkung links: \xmlflush{#1}}
		}
	\doif{\xmlatt{#1}{place}}{mTop}
		{
			\FSBfootnote{Anmerkung oben: \xmlflush{#1}}
		}
	\doif{\xmlatt{#1}{place}}{inline}
		{
			\FSBfootnote{Anmerkung in der Zeile: \xmlflush{#1}}
		}				
\stopxmlsetups

%% Innerhalb eines Segmentes kann sich ..
\startxmlsetups xml:tei:seg
	%% .. eine Sachanmerkung ..
	\doif{\xmlatt{#1}{type}}{comment} %% geändert Althof - orig: \doif{\xmlatt{#1}{type}}{sachapparat}
		 {\FSBsachapparat{%
			%% .. (dabei wird im Text, wenn Referenzen exportiert werden, ..
			\doifmodeelse{FSBreferenzen}
				{% .. der Bezug gedruckt und die Referenz entsprechend exportiert, ..
				\FSBexportelementline{note}{\FSBattribute{xml:id}{\xmlattribute{#1}{./note}{xml:id}}}{\xmlcommand{#1}{./orig}{xml:tei:seg:orig}}%
				}%
				{% .. sonst wird nur der Text gesetzt) ..
				\xmlcommand{#1}{./orig}{xml:tei:seg:orig}%
				}%
			}
			{\xmlcommand{#1}{./note}{xml:tei:seg:note}}
		}%
	%% .. oder eine Textanmerkung befinden.
	\doif{\xmlatt{#1}{type}}{critical} %% geändert Althof - orig: \doif{\xmlatt{#1}{type}}{textapparat}
		{
		%% Version im Sachapparat
			%\FSBsachapparat{\xmlcommand{#1}{./orig}{xml:tei:seg:orig}}
			%{\xmlcommand{#1}{./orig}{xml:tei:seg:orig}}{\xmlcommand{#1}{./note}{xml:tei:seg:note}}{} %% geändert Althof, weil Ausgabe von note in Text
			%{\xmlcommand{#1}{./note}{xml:tei:seg:note}}

			%% Version im Textapparat
			\FSBtextapparat{\xmlcommand{#1}{./orig}{xml:tei:seg:orig}}
				{\xmlcommand{#1}{./orig}{xml:tei:seg:orig}}{\xmlcommand{#1}{./note}{xml:tei:seg:note}}{}
		}%
\stopxmlsetups

%% Der Text, auf den sich ein Apparateintrag bezieht wird gedruckt.
\startxmlsetups xml:tei:seg:orig
	\xmlstrip{#1}{.}
	\xmlflush{#1}
\stopxmlsetups

%% Die Bemerkung eines Apparateintrages wird gedruckt.
\startxmlsetups xml:tei:seg:note
	\xmlflush{#1}%
\stopxmlsetups

%% Bei einer Ersetzung wird es komplizierter.
\startxmlsetups xml:tei:subst
	\enablemode[subst]
	%% Falls bei der Einfügung ein Ort angegeben ist, ..
	\xmldoifelse{#1}{./add[@place]}
		{% .. wird dieser im Textapparat erwähnt, ..
		\FSBtextapparat{\xmlcommand{#1}{./add}{xml:tei:subst:add}}
			{\xmlcommand{#1}{./add}{xml:tei:subst:add}}{\xmlattribute{#1}{./add}{place}}{\xmlcommand{#1}{./del}{xml:tei:subst:del}}
		}
		{% .. sonst heißt es nur 'korr. aus'.
		\FSBtextapparat{\xmlcommand{#1}{./add}{xml:tei:subst:add}}
			{\xmlcommand{#1}{./add}{xml:tei:subst:add}}{korr. aus}{\xmlcommand{#1}{./del}{xml:tei:subst:del}}
		}%
	\disablemode[subst]
\stopxmlsetups

%% Das Hinzugefügte einer Ersetzung wird ohne führende und folgende Leerzeichen gedruckt.
\startxmlsetups xml:tei:subst:add
	\ignorespaces\xmlflush{#1}\removeunwantedspaces%
\stopxmlsetups

%% Das Gestrichene einer Ersetzung wird, wenn es nicht überschrieben wurde, in spitzen Klammern gedruckt.
\startxmlsetups xml:tei:subst:del
	\doifelse{\xmlatt{#1}{rend}}{überschrieben}
		{\xmlflush{#1}}
		{\FSBlang\xmlflush{#1}\FSBrang}
\stopxmlsetups

%% Fehlende Worte werden ..
\startxmlsetups xml:tei:supplied
	%% .. bei Unsicherheit im Textapparat vermerkt, ..
	\doif{\xmlattribute{#1}{.}{cert}}{low}
		{\FSBtextapparat{}
			{\FSBwikitotex{\xmlatt{#1}{prev}}}{zu ergänzen wohl}{\xmlflush{#1}}
		}%
	%% .. ansonsten in eckigen Klammern dem Text hinzugefügt.
	\doif{\xmlattribute{#1}{.}{cert}}{high}
		{[\bgroup\it \xmlflush{#1}\egroup]}%
\stopxmlsetups

%% Überflüssige Wörter werden ..
\startxmlsetups xml:tei:surplus
	%% .. mit vorangehendem oder ..
	\xmldoif{#1}{.[@prev]}
		{\FSBtextapparat{}
			{\FSBwikitotex{\xmlatt{#1}{prev}}}{folgt}{\FSBlang\FSBlang\ignorespaces\xmlflush{#1}\removeunwantedspaces\FSBrang\FSBrang}%
		}%
	%% .. folgendem Wort als Lemma in den Textapparat geschrieben.
	\xmldoif{#1}{.[@next]}
		{\FSBtextapparat{}
			{\FSBwikitotex{\xmlatt{#1}{next}}}{davor}{\FSBlang\FSBlang\ignorespaces\xmlflush{#1}\removeunwantedspaces\FSBrang\FSBrang}%
		}%
\stopxmlsetups

%% ##################################################################
%% ##                      Register und Verweise                   ##
%% ##################################################################

%% Bei den Ankern werden, wenn gewünscht, die Zeilennummern exportiert.
\startxmlsetups xml:tei:anchor
	\xmldoif{#1}{.[@type='indexbereich']}
		{\removeunwantedspaces}%
	\doifmode{FSBreferenzen}
		{%
		%% Falls der Anker im Sachapparat auftaucht, ..
		\doifmodeelse{FSBinSachapparat}
			%% .. wird er entsprechend exportiert.
			{\FSBexportelementlineapparatus{anchor}{\FSBattribute{xml:id}{\xmlatt{#1}{xml:id}}}{}%
			}
			%% Ansonsten wird die normale Routine aufgerufen..
			%% .. wenn er sich nicht im Lemma befindet.
			{\doifmodeelse{inLemma}{}%
				{\FSBexportelementline{anchor}{\FSBattribute{xml:id}{\xmlatt{#1}{xml:id}}}{}}%
			}
		}%
\stopxmlsetups

%% Bei Referenzierungen wird, wenn gewünscht und vorhanden, das zugehörige Zitat ausgegeben.
\startxmlsetups xml:tei:ref
	\doifmode{FSBreferenzen}
		{\FSBgetextref{\FSBgetpath{\xmlatt{#1}{target}}}{\FSBgetref{\xmlatt{#1}{target}}}%
		}%
\stopxmlsetups

%% Bei Registereinträgen wird, wenn gewünscht, auch die Zeilennummern exportiert.
\startxmlsetups xml:tei:index
	\doifmode{FSBreferenzen}
		{%
		%% Falls der Eintrag im Sachapparat steht, .. Prüfung, damit in Register Eintrag kursiv gesetzt werden kann
		\doifmodeelse{FSBinSachapparat}
			%% wird er entsprechend exportiert.
			{\FSBexportelementlineapparatus{index}
				{\FSBattribute{spanTo}{\xmlatt{#1}{spanTo}}\space
				\FSBattribute{indexName}{\xmlatt{#1}{indexName}}\space
				\FSBattribute{corresp}{\xmlatt{#1}{corresp}}\space
				\FSBattribute{term}{\xmltext{#1}{./term}}\space
				}
				{}%
			}
			%% Ansonsten wird dies ..
			%% .. wenn sich der Eintrag nicht im Lemma befindet ..
			{\doifmodeelse{inLemma}{}%
				{\FSBexportelementline{index} %% schreibt in dei externe Datei die Daten für die Referenz
					{\FSBattribute{spanTo}{\xmlatt{#1}{spanTo}}\space %% indes hat das Attribut spanTo
					\FSBattribute{indexName}{\xmlatt{#1}{indexName}}\space %% ... hat das Attribut indexName etc.
					\FSBattribute{corresp}{\xmlatt{#1}{corresp}}\space
					\FSBattribute{term}{\xmltext{#1}{./term}}\space
					%% .. anders vermerkt.
					\FSBattribute{location}{text}
					}
					{}}%
			}%
		}%
	\ignorespaces%
	%\removeunwantedspaces%
	\doifmode{register}
		{\FSBortsregister[\xmltext{#1}{./term}]{\xmltext{#1}{./term}}}%
	\startFSBortsregister[\xmlall{#1}{term}]{\xmlall{#1}{term}} *** TODO *****************************************
	INDEX
	\xmlflush{#1}
	\FSBortsregister[\xmlall{#1}{term}]{TEST2}{TEST\xmlflush{#1}}
	\stopFSBortsregister
\stopxmlsetups


\stopenvironment

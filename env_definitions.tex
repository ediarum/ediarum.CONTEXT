\startenvironment env_definitions % Diese Datei enthälte alle dateiübergreifenden Definitionen
\project project_ediarum_context

%% ##################################################################
%% ##                      lua-Funktionen                          ##
%% ##################################################################

%% externe Dateien
\def\FSBfwrite{\LUAwritefile} % schreibt in die gesetzte Datei
\def\FSBfwriteN{\LUAwritefilelf} % schreibt einen Zeilenumbruch in die gesetzte Datei
\def\FSBfwriteT{\LUAwritefiletab} % schreibt einen Tabulator in die gesetzte Datei
\def\FSBnewfile{\LUAnewfile} % erstellt eine neue Datei als zusätzlichen Output

%% Referenzen
\def\FSBgetextfileid{\LUAgetextfileid} % gibt die aktuelle DateiID aus
\def\FSBgetextref{\LUAgetextref} % gibt zur angegebenen Datei und Element das Zitat aus
\def\FSBgetpath{\LUAgetpath} % gibt nur die Pfadangabe der Referenz aus
\def\FSBgetref{\LUAgetref} % gibt nur das Element der Referenz aus
\def\FSBsetextfileref{\LUAsetextfileref} % setzt zu einer Datei das Zitat fest
\def\FSBsetextref{\LUAsetextref} % setzt zu einem Element das Zitat fest

%% sonstiges
\def\FSBwikitotex{\LUAwikitotex} % interpretiert Wikischreibweise in ConTeXt

%% Zeilennummerierung
\def\FSBlinerefno{\LUAlinerefno} % definiert einen Zähler für Zeilenmarkierungen
\def\FSBlinesetup{\LUAlinesetup} % bestimmt wie mit zeilen- und seitenumfassenden Zitatspannen umgegangen wird
\def\FSBnextlineref{\LUAnextlineref} % erhöht den Zähler für Zeilenmarkierungen
\def\FSBpagenumber{\LUApagenumber} % gibt die Seitenzahl der Referenz wieder
\def\FSBstartlinerefebene{\LUAstartlinerefebene} % startet eine neue Verschachtelungsebene
\def\FSBstoplinerefebene{\LUAstoplinerefebene} % beendet eine Verschachtelungsebene

%% ##################################################################
%% ##                          Setups                              ##
%% ##################################################################

%% Schriften
\def\FSBgesperrt{\SETgesperrt} % schreibt gesperrt

%% Sonderzeichen
\def\FSBlang{\SETlang} % linke eckige Klammer
\def\FSBrang{\SETrang} % rechte eckige Klammer
\def\FSBlfloor{\SETlfloor} % Winkel links unten
\def\FSBrceil{\SETrceil} % Winkeln rechts oben

%% Sprachen
\def\FSBgreek{\SETgreek} % für griechische Schrift

%% Paragraphen
\def\FSBzweispalten{\SETzweispalten} % Zweispaltige Paragraphen

%% Kopfzeile
\definemarking[FSBbriefnummer] % zur späteren Referenzierung der Briefnummer
\definemarking[FSBbriefdatum] % zur späteren Referenzierung des Briefdatums
\definemarking[FSBnotebooktitle] % zur späteren Referenzierung des Titels (hinzu Althof)

%% Zeilennummerierung
\definelinenumbering[FSBdefaultlinenumbering] % die normale Zeilennummerierung
\def\FSBlinenumbers{\SETlinenumbers} % gibt die Zeilen der Referenz zurück
\def\FSBresetlinenumbering{\SETresetlinenumbering} % setzt die Zeilenzählung auf Null zurück

%% Apparate
\def\FSBtextapparat{\SETtextapparat} % schreibt in den Text und mit Zeilenreferenz und Lemma in den Textapparat
\def\FSBsachapparat{\SETsachapparat} % schreibt in den Text und mit Zeilenreferenz in den Sachapparat
\def\FSBfootnote{\SETfootnote} % schreibt eine Fußnote (hinzu Althof)
\def\FSBleer{\SETleer} % Befehl der die Fußnotennummern löscht
\def\FSBzeile{\SETzeile} % setzt eine Zeilenmarkierung zur späteren Referenzierung
\definenote[FSBdofootnote] % schreibte eine Autorenfußnote (hinzu Althof)
\definenote[FSBdotextapparat] % schreibt in den Textapparat
\definenote[FSBdosachapparat] % schreibt in den Sachapparat

%% Referenzen
\def\FSBattribute{\SETattribute} % Vorbereitung der Attribute für den Export
\def\FSBexportelementline{\SETexportelementline} % Export der Zeilen-ID-Konkordanz für Verweisziele
\def\FSBexportelementlineapparatus{\SETexportelementlineapparatus} % Export der Zeilen-ID-Konkordanz für Verweisziele im Apparat
\def\FSBstartreferenzen{\SETstartreferenzen} % startet den Modus für die Verwendung von Referenzen
\def\FSBstopreferenzen{\SETstopreferenzen} % stoppt den Modus für die Verwendung von Referenzen

%% Register
\definedescription[FSBregistereintrag]
\definedescription[FSBregistersubeintrag]
\definedescription[FSBregistersubsubeintrag]

\defineregister[FSBortsregister] % definiert ein Ortsregister

%% ##################################################################
%% ##                           Modi                               ##
%% ##################################################################

%% referenzen % Liest externe Referenzen ein und aktiviert referenzexport
%% FSBreferenzen % Interner Modus für den Referenzexport in eine Datei und die Verwendung
%% deckblatt % Erzeugt ein Deckblatt mit den Metadaten
\enablemode[deckblatt]
%% FSBlinebreak % Interner Modus für die Behandlung von Zeilenumbrüchen
%% register % Erzeugt die verschiedenen Register
%% FSBinSachapparat % Ist innerhalb eines Sachapparateintrags aktiviert.

\stopenvironment

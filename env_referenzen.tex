\startenvironment env_referenzen % Zum Einlesen der Referenzen
\project project_ediarum_context

%% ##################################################################
%% ##                       Basiselemente                          ##
%% ##################################################################

%% Generelle Registrierung des xml-Markups
\startxmlsetups xml:refs:setups
	%% Dabei muss darauf geachtet werden, dass sich die Bestimmungen hier nicht mit dem eigentlichen Transformationsrichtlinien überschneiden.
	\xmlsetsetup{refs}{*}{-}
	\xmlsetsetup{refs}{refs}{xml:refs:root}
	\xmlsetsetup{refs}{refs/file}{xml:refs:file}
	\xmlsetsetup{refs}{refs/file/(anchor|index|note)}{xml:refs:file:*}
\stopxmlsetups

%% Das Setup muss registriert werden
\xmlregisterdocumentsetup{refs}{xml:refs:setups}

%% ##################################################################
%% ##                          Elemente                            ##
%% ##################################################################

%% Die Wurzel wird weiterverarbeitet
\startxmlsetups xml:refs:root
	\xmlflush{#1}
\stopxmlsetups

%% Für alle Dateieinträge ..
\startxmlsetups xml:refs:file
	%% .. wird die ID gelesen und das Zitat der Datei vorbereitet.
	\FSBsetextfileref{\xmlatt{#1}{xml:id}}{\xmlatt{#1}{type} \xmlatt{#1}{idno}}%
	%% Dann kommen die Elemente.
	\xmlflush{#1}%
\stopxmlsetups

%% Für alle Anker ..
\startxmlsetups xml:refs:file:anchor
	%% .. wird das entsprechende Zitat vorbereitet.
	\FSBsetextref{\FSBgetextfileid}{\xmlatt{#1}{xml:id}}{\xmlatt{#1}{line}}%
\stopxmlsetups

%% Das gleiche gilt für alle Indizierungen ..
\startxmlsetups xml:refs:file:index
	\FSBsetextref{\FSBgetextfileid}{\xmlatt{#1}{xml:id}}{\xmlatt{#1}{line}}%
\stopxmlsetups

%% .. und Noten.
\startxmlsetups xml:refs:file:note
	\FSBsetextref{\FSBgetextfileid}{\xmlatt{#1}{xml:id}}{\xmlatt{#1}{line}}%
\stopxmlsetups

\stopenvironment

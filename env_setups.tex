\startenvironment env_setups % Die Setups
\project project_ediarum_context

%% ##################################################################
%% ##                     Externe Dateien                          ##
%% ##################################################################

%% Hier wird definiert, wo \product und \component nach TeX input files suchen sollen.
\usepath[{ethik,Vorlesungen}]
\usesubpath[{../xml}]

%% Ebenso wird hier definiert, wo nach Bildern gesucht werden soll.
\doifmodeelse{*product}
	{\setupexternalfigures[directory={../general_img}]}
	{\setupexternalfigures[directory={../general_img, img}]}

%% ##################################################################
%% ##                         Schriften                            ##
%% ##################################################################

%% Die Module müssen evtl. bei Formeln benutzt werden, sind hier aber überflüssig.
% \usemodule[simplefonts]
% \usemodule[fnt-25]

%% Das ist eine Definition einer Schrift ohne Kerning und Ligaturen für gesperrten Schriftsatz, ..
\definefont[SETserifkerning][Serif*none]
%% .. der hier definiert wird mit einer Methode, die LuaTex nutzt.
%\def\SETgesperrt#1{{\SETserifkerning\setuptolerance[space]\kerncharacters[0.2]#1}}
\def\SETgesperrt#1{{\SETserifkerning\kerncharacters[0.2]#1}}

%% Alle notwendigen Systemschriften werden für die Benutzung definiert, ..
\setupbodyfont[gentium] %.. zunächst Gentium, die griechische Zeichen enthält, ..
\setupbodyfont[modern] % .. dann als vorläufigen Hauptfont Modern.
%% Dies geschieht eigentlich über:
% \setmainfont[garamond] *** TODO ***************************************************************************
% \setupbodyfont[garamond]
% \definefontsynonym[Serif][Garamond]
	% Wobei die Schrift der Schleiermacherausgabe eigentlich Sabon ist.

%% Die Schriftgröße wird auf 11 gesetzt.
\setupbodyfont[11pt]

%% ##################################################################
%% ##                      Sonderzeichen                           ##
%% ##################################################################

%% Die verwendeten Sonderzeichen sind ..
\def\SETlang{$\langle$}%〈} % .. die stumpfe Klammer links ..
\def\SETrang{$\rangle$}%〉} % .. und rechts, ..
\def\SETlfloor{$\lfloor$} % .. der Winkel links unten ..
\def\SETrceil{$\rceil$} % .. und rechts oben.

%% ##################################################################
%% ##                        Seitenlayout                          ##
%% ##################################################################

%% Die Seitengröße für den Druck wird zunächst definiert, ..
\definepapersize[SETdruckedition][width=159mm,height=238mm]
%% .. und dann auf DIN A4 gedruckt.
\setuppapersize[SETdruckedition][A4]%[SETdruckedition]

%% Auf eine Seitennummerierung wird zunächst verzichtet, aber nicht auf ein 2-seitiges Layout.
\setuppagenumbering[alternative=doublesided,location=]

%% Schließlich wird festgelegt, dass ..
 \setuplayout[location=middle, % .. der Druck mittig auf der Seite erscheinen soll, ..
	width=112mm, % .. weiterhin werden die Textbreite, ..
	topspace=17mm, % .. die Höhe über der Kopfzeile, ..
	bottomspace=34mm, % .. die Höhe unter der Fußzeile, ..
	headerdistance=5mm, % .. der Abstand zwischen Kopfzeile und Text, ..
	header=3mm, % .. die Höhe der Kopfzeile ..
	footer=0mm, %.. die Höhe der Fußzeile ..
	height=187mm, % .. die Gesamthöhe von Kopfzeile bis Fußzeile, ..
	backspace=20mm, % .. sowie der Abstand zwischen Innenrand und Text definiert, ..
	marking=on] % .. weiterhin soll die Druckränder auf der Druckseite sichtbar sein.

%% Mit diesem Befehl können die unsichtbaren Ränder von Elementen sichtbar gemacht werden:
% \showstruts
% \showframe

%% ##################################################################
%% ##                         Sprachen                             ##
%% ##################################################################

%% Für den Satz und die Trennung wird die Sprache angegeben.
\mainlanguage[de]
\setuplanguage[de][patterns={de,agr}]

%% Ein Schalter für griechische Worte wird eingeführt.
\def\SETgreek#1{\language[agr]\gentium #1}

%% ##################################################################
%% ##                       Paragraphen                            ##
%% ##################################################################

%% Es wird ein mittlerer Einzug definiert, der zunächst abgeschaltet ist.
\setupindenting[no,medium,first]
%% Der Wortabstand wird als tolerant gesetzt, damit es keine überstehenden Zeilenenden gibt.
\setuptolerance[horizontal,tolerant]
%% Für schmale Passagen wird der Randabstand definiert.
% \setupnarrower[left=6mm]

%% Hier wird ein mehrspaltiges Layout definiert.
% \defineparagraphs[SETzweispalten]
	% [n=2, % Es hat zwei Spalten ..
	% before=, after=, distance=4mm] % .. mit eine kleinen Abstand zueinander.
%% Die erste dieser Spalten ..
% \setupparagraphs[SETzweispalten][1]
	% [width=54mm,align=left,style=it] % .. wird hier mit Breite, Ausrichtung und Textstyle definiert.
%% Das Layout kann aufgerufen werden, indem man schreibt
%% \FSBzweispalten Text der Spalte 1\\ Text der Spalte 2\\

%% ##################################################################
%% ##                        Kopfzeile                             ##
%% ##################################################################

%% Die Kopfzeile wird eingerichtet.
\setupheader[text]
	[after=]%,style={\switchtobodyfont[10pt]}]

%% Hier wird der Inhalt der Kopfzeile definiert.
\setupheadertexts[\setups{SETrightpage}][][][\setups{SETleftpage}]

%% Um auf die notwendigen Informationen zurückgreifen zu können, müssen diese mit /marking gesetzt worden sein.
%% Dabei stehen auf der linken Seite ..
\startsetups[SETrightpage]
	% .. links die Seitenzahl ..
	\rlap{\bgroup\switchtobodyfont[10pt] \pagenumber\egroup}
	\hfill
	% .. und in der Mitte die aktuelle Briefnummer.
	%\expanded{\bgroup\switchtobodyfont[10pt] \it \getmarking[FSBbriefnummer][current]\egroup}# % origianal
	\expanded{\bgroup\switchtobodyfont[10pt] \it \getmarking[FSBnotebooktitle][current]\egroup} % geändert Althof für Gödel
	\hfill
	\llap{}
\stopsetups
%% Auf der rechten Seite stehen ..
\startsetups[SETleftpage]
	\rlap{}
	\hfill
	% .. in der Mitte das aktuelle Briefdatum ..
	\expanded{\bgroup\switchtobodyfont[10pt] \it \getmarking[FSBbriefnummer][current]\egroup}
	\hfill
	% .. und rechts die Seitenzahl.
	\llap{\bgroup\switchtobodyfont[10pt] \pagenumber\egroup}
\stopsetups
%% *** TODO **********************************************************************************************
%% Eigentlich sollen links in der Mitte die Briefnummern der Doppelseite, bzw. bei einem Brief noch in Klammern der Absender stehen.
%% Und rechts sollen in der Mitte die Datumsangaben bzw. die Datumsspanne der Briefe auf der Doppelseite stehen.

%% ##################################################################
%% ##                     Zeilennummerierung                       ##
%% ##################################################################

%% Zunächst wird die Standardnummerierung eingerichtet, ..
\setuplinenumbering[FSBdefaultlinenumbering]
	[location=inner, % .. sie befindet sich innen, ..
	step=5, % .. nummeriert jede fünte Zeile, ..
	method=text, % .. ??
	style={\switchtobodyfont[9pt]}, % .. der Zahlenstil wird festgelegt, ..
	align={\SETinner}, % .. diese sind nach außen ausgerichtet, ..
	distance=3mm, % .. haben einen kleinen Abstand zum Text,
	width=3mm] % .. und eine reserviert Breite.
%% Die Ausrichtung der Zeilennummern wechselt je nach Seite.
\def\SETinner{\doifoddpageelse{left}{right}}

%% Wenn Zeilen mit \startline[ID] und \stopline[ID] markiert worden sind, kann hiermit wieder auf sie verwiesen werden.
\def\SETlinenumbers#1{%
	{% Falls der Startanker gefunden wurde, der mit lr:b: beginnt, ..
	\doifreferencefoundelse{lr:b:#1}
		{% .. wird dieser in eine lokale Variable eingelesen.
		\edef\fline{\currentreferencelinenumber}%
		% Das gleiche passiert mit dem Stopanker.
		\doifreferencefoundelse{lr:e:#1}
			{\edef\tline{\currentreferencelinenumber}%
			% Und schließlich wird der Zeilenbereich gemäß dem Setup zurückgegeben.
			\FSBlinesetup{\FSBpagenumber{lr:b:#1}}{\fline}{\FSBpagenumber{lr:e:#1}}{\tline}
			}%
			{% Falls eine der Referenzen nicht gefunden wurde, gibt es eine Fehlermeldung ..
			\unknownreference{#1}% .. im Protokoll ..
			{UNB REF: 'lr:e:#1'}% .. und im Text.
			}%
		}%
		{\unknownreference{#1}{UNB REF: 'lr:b:#1'}}%
	}}

%% Hiermit wird die Zeilenzählung wieder auf Null gesetzt.
\def\SETresetlinenumbering[#1]{%
	\startlinenumbering[#1]%
	\stoplinenumbering%
	}

%% ##################################################################
%% ##                        Apparate                              ##
%% ##################################################################

%% Der Textapparat wird definiert.
\def\SETtextapparat#1#2#3#4{% 1:im Text 2:Lemma 3:Aussage 4:Wort
	% Falls kein eigener Textapparat gedruckt werden soll, weil es bereits einen gibt, wird nur der Text gedruckt.
	\doifmodeelse{keinTextapparat}
		{#1}
	% Falls es einen Apparateintrag zu einem Lemma gibt, wird dieser dort nur angehängt.
	{\doifmodeelse{inLemma}
		{#2] {\it\nospace #3} #4\enablemode[lemmaSet]}
	% Falls es einen Apparateintrag zu einem Apparateintrag gibt, wird dieser dort nur angehängt.
	{\doifmodeelse{inApparat}
		{#2 {\it #3} #4}
	% Falls keiner der obigen Fälle zutrifft, wird ein einfacher Textapparat erstellt.
		{% Zunächst wird sichergestellt, dass vorherige Leerzeichen nicht gelöscht werden.
		\ifhmode\kern\zeropoint\fi%
		% Dann wird die Referenznummer erhöht ..
		\FSBnextlineref{}%
		% .. und ein Textapparat ..
		\FSBdotextapparat{%
			% .. mit Zeilenreferenz, ..
			{\bf \FSBlinenumbers{\FSBlinerefno}}%
			% .. Lemma ..
			\enablemode[inLemma]%
			% .. (die abschließende Klammer des Lemmas wird nur gedruckt, wenn nicht noch ein anderes Lemma angefügt wurde) ..
			#2\doifmodeelse{lemmaSet}{}{]\disablemode[lemmaSet]}
			\disablemode[inLemma]%
			% .. und Apparateintrag wird erzeugt.
			\enablemode[inApparat]%
			{\it #3}
			#4
			\disablemode[inApparat]%
			}%
		% Schließlich wird der Textinhalt gedruckt, ..
		\FSBzeile{\enablemode[keinTextapparat]#1\disablemode[keinTextapparat]}%
		% .. oder, falls es keinen gibt, nicht.
		\doifempty{#1}{\nospace}%
		}%
	}}%
	}

%% Sehr viel kürzer ist der Sachapparat ohne Lemma.
\def\SETsachapparat#1#2{% 1:im Text 2:Aussage
	% Zunächst wird sichergestellt, dass vorherige Leerzeichen nicht gelöscht werden.
	\ifhmode\kern\zeropoint\fi%
	% Dann wird die Referenznummer erhöht ..
	\FSBnextlineref{}%
	% .. und ein Sachapparat ..
	\FSBdosachapparat{
		% .. mit Zeilenreferenz ..
		{\bf \FSBlinenumbers{\FSBlinerefno}}%
		% .. und kursivem Apparateintrag wird erzeugt.
		\enablemode[FSBinSachapparat]%
		{\it #2}%
		\disablemode[FSBinSachapparat]%
		}%
	% Schließlich wird der Textinhalt gedruckt.
	\FSBzeile{#1}%
	}

%% Für die Fußnoten des Autors
\def\SETfootnote#1{
	\FSBdofootnote{\it #1}
}

%% Um auf einen Zeilenbereich referenzieren zu können, wird dieser hier mit Ankern gekennzeichnet.
\def\SETzeile#1{%
	% Der Startanker der Zeilenreferenzierung wird gesetzt, intern heißt die Referenz 'lr:b:\FSBlinerefno'.
	\startline[\FSBlinerefno]{}%
	% Damit sich überschneidende Zeilenreferenzierungen und \FSBlinerefno nicht behindern, wird in eine andere Ebene gewechselt, ..
	\FSBstartlinerefebene%
	% .. bevor der Inhalt gedruckt wird ..
	#1%
	% .. und die Ebene wieder beendet wird.
	\FSBstoplinerefebene%
	% Der Stopanker der Zeilenreferenzierung wird gesetzt, intern heißt die Referenz 'lr:e:\FSBlinerefno'.
	\stopline[\FSBlinerefno]{}%
	}

%% Eine Definition, um die Fussnotenzahl zu unterdrücken. Man kann aber dies auch erreichen mit \footnote[-]{fussnote}.
\def\SETleer#1{}

%% Hier wird der Textapparat konfiguriert.
\setupnote[FSBdotextapparat]
	[rule=off, % Es gibt keine voreingestellte Linie.
	paragraph=yes, % Die Einträge werden in einem Absatz gedruckt.
	before={\bgroup\blank\hairline\vskip0.2em\egroup}, % Hier wird eine eigene Linie definiert.
	after=, % Es folgt kein Abstand o.ä.
	inbetween=\hskip5mm, % Zwischen den Einträgen ist ein kleiner Abstand.
	% numbercommand=\FSBleer, % Die Nummer in der Fußnote wird unterdrückt.
	textcommand=\FSBleer, % Die Nummer im Text wird unterdrückt.
	style={\switchtobodyfont[9pt]}] % Der Schriftstil wird festgelegt.
\setupnotation[FSBdotextapparat]
	[numbercommand=\FSBleer, % Die Nummer in der Fußnote wird unterdrückt.
	]
%% Der folgende Befehl soll die Einträge hintereinander anlegen, scheint aber überflüssig.
%%% \setupnotedefinition[FSBdotextapparat][location=serried]

%% Hier wird der Sachapparat konfiguriert.
\setupnote[FSBdosachapparat]
	[rule=off, % Es gibt keine voreingestellte Linie.
	paragraph=yes, % Die Einträge werden in einem Absatz gedruckt.
	before={\vskip0.5em\hl[5]\vskip0.5em}, % Hier wird eine eigene Linie definiert.
	after=, % Es folgt kein Abstand o.ä.
	inbetween=\hskip5mm, % Zwischen den Einträgen ist ein kleiner Abstand.
	% numbercommand=\FSBleer, % Die Nummer in der Fußnote wird unterdrückt.
	textcommand=\FSBleer, % Die Nummer im Text wird unterdrückt.
	style={\switchtobodyfont[9pt]}] % Der Schriftstil wird festgelegt.
\setupnotation[FSBdosachapparat]
	[numbercommand=\FSBleer, % Die Nummer in der Fußnote wird unterdrückt.
	]
%% Der folgende Befehl soll die Einträge hintereinander anlegen, scheint aber überflüssig.
% \setupnotedefinition[FSBdosachapparat][location=serried]

%% Die Fußnoten des Autors erscheinen als Fußnoten im Text.
\setupnote[FSBdofootnote]
	[rule=off,
	paragraph=yes, % Die Einträge werden in einem Absatz gedruckt.
	before={\blank\vskip5mm}, %{\vskip0.5em\hl[5]\vskip0.5em}, % Hier wird eine eigene Linie definiert.
	after=, % Es folgt kein Abstand o.ä.
	%before={\vskip3mm\startlinenumbering[FSBdefaultlinenumbering][continue]\noindenting}, % Vor den Noten gibt es einen kleinen Durchschuß, und die Zeilenzählung geht weiter ..
	%after={\stoplinenumbering}, % .. und endet hinterher.
	%location=text, % Die Noten werden im Text gedruckt.
	%bodyfont=, % Ohne veränderte Schrift.
	% numbercommand=\bf, % Die Nummer ist fett.
	% margindistance=0mm, % Zum Rand gibt es keinen Abstand ..
	style={\switchtobodyfont[9pt]}] % Der Schriftstil wird festgelegt.
	%********** TODO ***************+
	% Die Autorennote muss mit jedem Brief wieder von vorne zählen.
	% Evtl. setupnotes[n=1] ? Vgl. \setupfootnotes
\setupnotation[FSBdofootnote]
	[numbercommand=\bf, % Die Nummer ist fett.
	numberconversion=a, % Sie werden a, b, c nummeriert.
	alternative=left,
	margindistance=5mm, % Zum Rand gibt es einen Abstand ..
	after=\hskip0.5em, % Abstand zum nächsten Element
	]

%% Der Fußnoteneintrag ist links ausgerichtet und hat einen kleinen Abstand zur Nummer.
% \setupnotedefinition[FSBautornote][location=left,width=5mm]


%% ##################################################################
%% ##                         Referenzen                           ##
%% ##################################################################

\def\SETerror{UNB REF}
%% Hier werden die Attribute für den Export vorbereitet
\def\SETattribute#1#2{#1="#2"}

%% Hiermit kann die Zeilennummer eines Elementes für die Zeilen-ID-Konkordanz exportiert werden.
\def\SETexportelementline#1#2#3{% 1: Elementname 2:Attribute 3:Inhalt
	% Die Referenznummer wird erhöht ..
	\FSBnextlineref{}%
	% .. der Zeilenbereich gedruckt ..
	\FSBzeile{#3}%
	% .. wenn der Zeilenreferenzen gefunden wurden ..
	\doifreferencefoundelse{lr:b:\FSBlinerefno}
		{\edef\fline{\currentreferencelinenumber}%
		\doifreferencefoundelse{lr:e:\FSBlinerefno}
			{\edef\tline{\currentreferencelinenumber}%
			% .. wird das Element ..
			\FSBfwriteT%
			\FSBfwriteT%
			\FSBfwrite{<#1 #2 line="}%
			% .. mit der Zeilennummer exportiert.
			\FSBfwrite{\FSBlinesetup{\FSBpagenumber{lr:b:\FSBlinerefno}}{\fline}{\FSBpagenumber{lr:e:\FSBlinerefno}}{\tline}}%
			\FSBfwrite{"/>}%
			\FSBfwriteN%
			}
			% Falls die Referenzen nicht gefunden wurden passiert nichts.
			% Wenn hier ein zu lange Fehlermeldung auftauchte, störte dies die PDF-Erstellung; ebenso muß hier eine eigene Prüfung der Referenzen erfolgen, die direkte Nutzung von \FSBlinenumbers führt zu Fehlern.
			{}%
		}
		{}%
	}

%% Hiermit kann die Zeilennummer eines Elementes, das sich in einer Fussnote befindet, für die Zeilen-ID-Konkordanz exportiert werden.
\def\SETexportelementlineapparatus#1#2#3{% 1: Elementname 2:Attribute 3:Inhalt
	% Die Referenznummer wird NICHT erhöht (sondern die letzte des Apparateintrags genutzt) ..
	% .. der Zeilenbereich gedruckt ..
	#3%
	% .. wenn Zeilenreferenzen gefunden wurden ..
	\doifreferencefoundelse{lr:b:\FSBlinerefno}
		{\edef\fline{\currentreferencelinenumber}%
		\doifreferencefoundelse{lr:e:\FSBlinerefno}
			{\edef\tline{\currentreferencelinenumber}%
			% .. wird das Element ..
			\FSBfwriteT%
			\FSBfwriteT%
			\FSBfwrite{<#1 #2 line="}%
			% .. mit der Zeilennummer exportiert.
			\FSBfwrite{\FSBlinesetup{\FSBpagenumber{lr:b:\FSBlinerefno}}{\fline}{\FSBpagenumber{lr:e:\FSBlinerefno}}{\tline}}%
			\FSBfwrite{" location="apparatus"/>}%
			\FSBfwriteN%
			}
			% Falls die Referenzen nicht gefunden wurden passiert nichts.
			% Wenn hier ein zu lange Fehlermeldung auftauchte, störte dies die PDF-Erstellung; ebenso muß hier eine eigene Prüfung der Referenzen erfolgen, die direkte Nutzung von \FSBlinenumbers führt zu Fehlern.
			{}%
		}
		{}%
	}

%% Hier wird eine neue Datei für den Export der Zeilen-ID-Konkordanz angelegt und der Modus für den Export begonnen.
\def\SETstartreferenzen#1{%
	\enablemode[FSBreferenzen]%
	\FSBnewfile{lokaleReferenzen.xml}%
	\FSBfwrite{<?xml version="1.0" encoding="UTF-8"?>}%
	\FSBfwriteN%
	\FSBfwrite{<refs>}%
	\FSBfwriteN%
	}
%% Die Exportdatei und der Exportmodus werden beendet.
\def\SETstopreferenzen#1{%
	\FSBfwrite{</refs>}%
	\disablemode[FSBreferenzen]%
	}

%% ##################################################################
%% ##                        Register                              ## *** TODO ******************************
%% ##################################################################

\setupdescription[FSBregistereintrag]
	[alternative=top,
	inbetween=\nospace,%\vskip1mm,
	headstyle={\it},%\setupalign[flushleft,broad,nothyphenated]},
	headalign={flushleft},%broad},%nothyphenated},
	headcommand=\hskip-3mm,
	margin=3mm,
	textcommand=,
	after=,
	before=,
	]

\setupdescription[FSBregistersubeintrag]
	[alternative=top,
	inbetween=\nospace,%\vskip1mm,
	headstyle={\it},
	headcommand={\hskip-3mm\inframed[width=3mm,offset=0mm,align=flushleft,frame=off]{–}},
	margin=3mm,
	textcommand=,
	after=,
	before=,
	]

\setupdescription[FSBregistersubsubeintrag]
	[alternative=top,
	text=,
	inbetween=\nospace,%\vskip1mm,
	headstyle={\it},
	headcommand={\hskip-3mm\inframed[width=3mm,offset=0mm,align=flushleft,frame=off]{–}},
	margin=6mm,
	textcommand=,
	after=,
	before=,
	]
% \doifmode{test}{

%%%\def\SETdots#1{#1\dotfill}
%%%\def\SETcrlf#1{#1\crlf}
% \def\SETregistertext#1{\indenting[first]#1\crlf}
% \setupregister[FSBortsregister]
	% [n=2,
	% indicator=no,
	% textcommand=\SETregistertext,
	% distance=0mm]


% gute Befehle
% \hairline
% \blank
%\setupbackend[export=yes]
%\setupbackend[export=yes,xhtml=yes,css={context-export.css},]
	%\setupbackend[format=PDF/X-1a:2003,intent={ISO Coated v2 (ECI)}] % für PDF-A ??
% }


\stopenvironment

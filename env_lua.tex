\startenvironment env_lua % Der Lua-Code, der ConTeXtzugriff steht am Ende der Datei
\project project_ediarum_context

\startluacode
--% #################################################################
--% ##                    Externe Dateien                          ##
--% #################################################################

--% Die Variable beinhaltet den Namen der externen Datei.
outputfile = ''

--% Erstellt eine neue Datei und setzt sie als externe Datei.
--% 'filename' Der Name der neuen Datei
function newFile(filename)
	outputfile = filename
    local f = assert(io.open(outputfile, "w"))
    f:close()
end

--% Schreibt den String in die gesetzte Datei.
--% 'str' Der zu schreibende Text
function writeFile(str)
    local f = assert(io.open(outputfile, "a"))
    f:write(str)
    f:close()
end

--% Schreibt einen Zeilenumbruch in die gesetzte Datei.
function writeFileNewline()
    local f = assert(io.open(outputfile, "a"))
    f:write('\n')
    f:close()
end

--% Schreibt einen Tabulator in die gesetzte Datei.
function writeFileNewtab()
    local f = assert(io.open(outputfile, "a"))
    f:write('\t')
    f:close()
end

--% #################################################################
--% ##                        Referenzen                           ##
--% #################################################################

--% Die Variablen beinhalten ..
extfileID = '' --% .. die momentane DateiID
extreff = {} --% .. ein Array für den Verweis auf die Datei (mit 'content') und die einzelnen Referenzen.

--% Schreibt die momentane DateiID.
function getExtFileID ()
	context(extfileID)
end

--% Gibt die angeforderte Referenz mit Zeilennummern aus.
--% 'fileID' Die ID der das Verweisziel beinhaltenden Datei
--% 'elementID' Die ID des Verweiszieles
function getExtRef (fileID,elementID)
	--% Falls keine Datei mit der angebenen ID eingelesen wurde ..
	if extreff[fileID]==nil then
		--% .. wird eine Fehlermeldung zurückgegeben.
		context('Referenz nicht gefunden: Datei: "'..fileID..'", Element: "'..elementID..'"')
	--% Falls zwar die Datei, aber kein Element mit der ID eingelesen wurde, ..
	elseif extreff[fileID][elementID]==nil then
		--% .. wird auch eine entsprechende Fehlermeldung wiedergegeben.
		context('Referenz nicht gefunden: Element "'..elementID..'"')
	--% Wenn es sich bei dem angegebenen Element um einen Startanker handelt, ..
	elseif string.sub(elementID,1,5)=='start' then
		--% .. wird nach dem Stopanker gesucht.
		local stopElement='stop'..string.sub(elementID,6,-1)
		--% Existiert kein Stopanker, ..
		if extreff[fileID][stopElement]==nil then
			--% .. wird nur die Briefnummer und Zeile des Startankers ausgegeben, ..
			context(extreff[fileID]['content']..', '..extreff[fileID][elementID])
		else
			--% .. ansonsten wird der Verweis auf den Abschnitt eingefügt.
			context(extreff[fileID]['content']..', ')
			linesetup(1,extreff[fileID][elementID]+0,1,extreff[fileID][stopElement]+0)
		end
	--% Wenn es sich um ein anderes Element handelt, ..
	else
		--% .. wird der entsprechende Verweis eingefügt.
		context(extreff[fileID]['content']..', '..extreff[fileID][elementID])
	end
end

--% Liefert vom übergebenen String den Pfad vor dem ersten '/#' bzw. den ganzen String zurück.
--% 'ref' Der String mit einer Pfadangabe
function getPath (ref)
	if string.find(ref, '/#')==nil then
		context(ref)
	else
		local i,j = string.find(ref, '/#')
		context(string.sub(ref,1,i-1))
	end
end

--% Liefert die Referenz hinter einem '/#' bzw. den ganzen String zurück.
function getRef (ref)
	if string.find(ref, '/#')==nil then
		context(ref)
	else
		local i,j = string.find(ref, '/#')
		context(string.sub(ref,j+1,-1))
	end
end

--% Setzt eine neue DateiID fest und schreibt die Zitierweise in das Array extreff.
--% 'fileID' Die neue DateiID
--% 'content' Die Zitat zu der Datei
function setExtFileRef (fileID,content)
	--% Zu der DateiID wird ein neues Subarray erzeugt ..
	extreff[fileID] = {}
	--% Die DateiID wird als aktuelle festgelegt ..
	extfileID = fileID
	--% .. und die Zitierweise wird als content in das Array geschrieben.
	extreff[fileID]['content'] = content
end

--% Schreibt eine Zitat für ein bestimmtes Element in das Array extreff.
--% 'fileID' Die zum Zitat gehörige Datei
--% 'elementID' Das zum Zitat gehörige Element
--% 'content' Das Zitat
function setExtRef (fileID,elementID,content)
	extreff[fileID][elementID]=content
end

--% #################################################################
--% ##                         Sonstiges                           ##
--% #################################################################

--% Interpretiert Wikischreibweise in ConTeXt.
--% 'str' Der zu interpretierende String
function wikitotex(str)
	--% Unterstrichener, kursiver und fetter Modus sind nicht gesetzt ..
	ul, it, bf = 0, 0, 0
	--% .. und bisher wurde nichts geschrieben.
	strtex = ''
	--% Für jeden Buchstaben des Strings ..
	for i=1, string.len(str) do
		ch = string.sub(str, i, i)
		--% .. wird geprüft, ob es ein Unterstrich ist ..
		if ch=='_' then
			--% .. (das hieße, wenn der Kursivmodus noch nicht gesetzt ist, ..
			if ul==0 then
				--% .. daß ab sofort alles kursiv gesetzt wird, ..
				strtex=strtex..'\\FSBgesperrt{ '
				ul=1
			else
				--% .. und daß der Modus sonst beendet wird).
				strtex=strtex..'}'
				ul=0
			end
		--% Genauso wird geprüft, ob es sich bei dem Zeichen um ein Sternchen handelt, ..
		elseif ch=='*' then
			--% .. mit dem für den fetten Modus in obiger Weise verfahren wird, ..
			if bf==0 then
				strtex=strtex..'{\\bf '
				bf=1
			else
				strtex=strtex..'}'
				bf=0
			end
		--% .. oder um einen Slash, ..
		elseif ch=='/' then
			--% der für den kursiven Modus steht.
			if it==0 then
				strtex=strtex..'{\\it '
				it=1
			else
				strtex=strtex..'}'
				it=0
			end
		--% Ist es ein anderes Zeichen ..
		else
			-- .. wird es einfach an die Ausgabe angehängt.
			strtex=strtex..ch
		end
	end
	--% Die fertige Ausgabe wird schließlich mit ConTeXt gedruckt.
	context(strtex)
end

--% #################################################################
--% ##                     Zeilennummerierung                      ##
--% #################################################################

--% Die Variablen sind ..
linerefebene = 1 --% .. die Tiefe der Verschachtelung von Zeilenreferenzen, ..
linerefno = 0 --% .. die Nummer der Zeilenreferenz ..
linerefnokorr = {0} --% .. und die Korrektur für jede Ebene als Array.

--% Die aktuelle Zeilenreferenznummer wird ausgegeben, dabei wird eine evtl. Ebenenkorrektur berücksichtigt.
function getLinerefno()
		context(linerefno-linerefnokorr[linerefebene])
end

--% Der Umgang mit Zeilen- und Seitenspannen wird hier konfiguriert und die Spanne entsprechend ausgegeben.
--% 'page1' Die Seitennummer am Anfang der Spanne
--% 'line1' Die Zeilennummer am Anfang der Spanne
--% 'page2' Die Seitennummer am Ende der Spanne
--% 'line2' Die Zeilennummer am Ende der Spanne
function linesetup(page1,line1,page2,line2)
	--% Da hier die Zeilen pro Brief gezählt werden, ist keine Unterscheidung nach Seitenzahl notwendig.
--	if page1==page2 then
		--% Falls es der Abschnitt in der selben Zeile endet, wie er beginnt, ..
		if line1==line2 then
			--% .. wird diese Zeile ausgegeben.
			tex.sprint(line1)
		--% Endet der Abschnitt eine Zeile nach Beginn, ..
		elseif line1+1==line2 then
			--% .. wird hinter einem kleinen Abstand hinter der Zeilennummer ein 'f' ergänzt.
			tex.sprint(line1..' f')
		--% Bei einem längeren Abschnitt ..
		else
			--% .. werden Anfang und Ende durch einen Strich verbunden.
			tex.sprint(line1..'–'..line2)
		end
--	else
--		tex.sprint(line1..'–S.'..page2..'.'..line2)
--	end
end

--% Die nächste Zeilenreferenznummer wird aufgerufen.
function nextLinerefno()
	--% Der Zähler wird um eins erhöht.
	linerefno = linerefno +1
	--% Für alle Verschachtelungsebenen unter dieser ..
	for i=1,linerefebene do
		--% wird die Korrektur um eins erhöht.
		linerefnokorr[i] = linerefnokorr[i] +1
	end
	--% Die Korrektur auf der aktuellen Ebene wiederum fällt weg.
	linerefnokorr[linerefebene] = 0
end

--% Die Seitenzahl der Referenz wird ausgegeben.
function pagenumber(ref_one)
	local one, bug = structures.references.identify("",ref_one)
	if bug then
--		tex.sprint("[unknown reference: %s]",ref_one)
--		return
	end
	structures.references.analyze(one)
	tex.sprint(one.realpage)
end

--% Es wird in ein höhere Verschachtelungsebene gewechselt.
function startLinerefebene()
	--% Der Zähler wird entsprechend erhöht.
	linerefebene = linerefebene +1
	--% In der neuen Ebene ist noch keine Korrektur nötig.
	linerefnokorr[linerefebene] = 0
end

--% Es wird wieder in eine tiefere Verschachtelungsebene gewechselt.
function stopLinerefebene()
	linerefebene = linerefebene -1
end

\stopluacode

%% ##################################################################
%% ##                       Definitionen                           ##
%% ##################################################################

%% Externe Dateien
\def\LUAnewfile#1{\ctxlua{newFile('#1')}} % filename % erstellt eine neue Datei als zusätzlichen Output
\def\LUAwritefile#1{\ctxlua{writeFile('#1')}} % text % schreibt in die gesetzte Datei
\def\LUAwritefilelf{\ctxlua{writeFileNewline()}} % schreibt einen Zeilenumbruch in die gesetzte Datei
\def\LUAwritefiletab{\ctxlua{writeFileNewtab()}} % schreibt einen Tabulator in die gesetzte Datei

%% Referenzen
\def\LUAgetextfileid{\ctxlua{getExtFileID()}} % gibt die aktuelle DateiID aus
\def\LUAgetextref#1#2{\ctxlua{getExtRef('#1','#2')}} % fileID elementID % gibt zur angegebenen Datei und Element das Zitat aus
\def\LUAgetpath#1{\ctxlua{getPath('#1')}} % reference % gibt nur die Pfadangabe der Referenz aus
\def\LUAgetref#1{\ctxlua{getRef('#1')}} % reference % gibt nur das Element der Referenz aus
\def\LUAsetextfileref#1#2{\ctxlua{setExtFileRef('#1','#2')}} % fileID content % setzt zu einer Datei das Zitat fest
\def\LUAsetextref#1#2#3{\ctxlua{setExtRef('#1','#2','#3')}} % fileID elementID content % setzt zu einem Element das Zitat fest

%% Sonstiges
\def\LUAwikitotex#1{\ctxlua{wikitotex("#1")}} % text % interpretiert Wikischreibweise in ConTeXt

%% Zeilennummerierung
\def\LUAlinerefno{\ctxlua{getLinerefno()}} % definiert einen Zähler für Zeilenmarkierungen
\def\LUAlinesetup#1#2#3#4{\ctxlua{linesetup(#1,#2,#3,#4)}} % startpage startline stoppage stopline % bestimmt wie mit zeilen- und seitenumfassenden Zitatspannen umgegangen wird
\def\LUAnextlineref{\ctxlua{nextLinerefno()}} % erhöht den Zähler für Zeilenmarkierungen
\def\LUApagenumber#1{\ctxlua{pagenumber("#1")}} % refnr % gibt die Seitenzahl der Referenz wieder
\def\LUAstartlinerefebene{\ctxlua{startLinerefebene()}} % startet eine neue Verschachtelungsebene
\def\LUAstoplinerefebene{\ctxlua{stopLinerefebene()}} % beendet eine Verschachtelungsebene

\stopenvironment
